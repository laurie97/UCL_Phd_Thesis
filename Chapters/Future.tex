\chapter{Future Prospects of Di-$b$-jet Searches}
\label{sec:fut}

This chapter will consider the future prospects of di-$b$-jet searches at ATLAS
and suggest possible developments of the analysis presented in this thesis
that can be considered in the future.

\section{Di-$b$-jet searches with Increased Luminosity}

The LHC has been collecting 13~TeV $pp$ collision data since January 2015 and is scheduled to continue to until 2038~\cite{fut-lhc-shedule}.
The di-$b$-jet searches presented in Chapters~\ref{sec:evt}-\ref{sec:lim} used 13~TeV $pp$ collision data collected in 2015 and 2016.
%From 2015-2022 it is expected that $pp$ collision data at a similar instantaneous luminosity as was collected in 2015 and 2016;
$pp$ collision data with an integrated luminosity of $\sim$300~\ifb{} is expected to be collected by the end of 2022.
After 2022, significant upgrades to the LHC accelerator and ATLAS detector are planned such that data can be taken at a higher instantaneous luminosity,
this is known as the High-Luminosity LHC.
The High-Luminosity LHC is expected to collect 13~TeV $pp$ collision data with an integrated luminosity of $\sim$3000~\ifb{}

Table~\ref{tab:fut-lumi} outlines the integrated luminosity of di-$b$-jet searches performed
using 13~TeV $pp$ collisions and the expected luminosities at key points in the LHC schedule.
The table includes the \lm{} and \hm{} data-set analysis that are soon to be published together.

{\renewcommand{\arraystretch}{1.2}
\begin{table}[!htb]
\centering
\begin{tabular}{|c|r l|r l|}
  \hline
End of Data             & \multicolumn{2}{c|}{Integrated Luminosity}         &  \multicolumn{2}{c|}{Integrated Luminosity}  \\
Collection     &  \multicolumn{2}{c|}{using a Single Jet Trigger}   &  \multicolumn{2}{c|}{using a Double $b$-Jet Trigger} \\
\hline
End of 2015             &  3.2 \ifb{} & \cite{dibjet-mori16_paper}                              & 3.2 \ifb{} & \cite{dibjet-lhcp_conf}                      \\
July  2016              & 13.1 \ifb{} & \cite{dibjet-ichep_conf} (\summer{})                    & \multicolumn{2}{c|}{No analysis performed}                   \\
End of 2016             & 36.1 \ifb{} & (\hm{})                              & 24.3 \ifb{} & (\lm{})              \\
End of 2022             & $\sim$ 300 \ifb{} &  \textit{(Projection)}          & $\sim$ 300  \ifb{} &  \textit{(Projection)}     \\
End of 2038             & $\sim$ 3000 \ifb{} & \textit{(Projection)}          & $\sim$ 3000 \ifb{} &  \textit{(Projection)}    \\
\hline
\end{tabular}
\caption[A summary of the integrated luminosity of data-sets used by the di-$b$-jet analyses performed at ATLAS and the expected integrated luminosities at key points in the LHC schedule]
        {A summary of the integrated luminosity of data-sets used by the di-$b$-jet analyses performed at ATLAS and the expected integrated luminosities at key points in the LHC schedule~\cite{fut-lhc-shedule}.
          All data-sets contain 13~TeV $pp$ collision data collected since January 2015,
          with the exception of the \lm{} data-set which is collected from January 2016.
          %Citations are given for published analyses.
          Di-$b$-jet analyses of the \summer{} and \lm{} data-set are both presented together in Chapters~\ref{sec:evt}-\ref{sec:lim} of this thesis.
          Di-$b$-jet analysis of the \hm{} data-set are soon to be published joint with the \lm{} data-set analysis.
        }
\label{tab:fut-lumi}
\end{table}}

The sensitivity to signal of the search phase in the di-$b$-jet analysis can be estimated as $\sqrt{S}/B$,
where $S$ and $B$ are the number of signal and background events passing the di-$b$-jet event selection in the mass region of the signal.
This approximation assumes that a perfect background estimation model is used, where no signal induced fit bias can occur.
Therefore the sensitivity of the di-$b$-jet analysis is proportional to the square root of the integrated luminosity.
Using this approximation and the values in Table~\ref{tab:fut-lumi}, it can be seen that di-$b$-jet searches performed on data-sets that include data collected in 2016
are more sensitive than equivalent searches only using data collected in 2015 by a factor of $\sim\sqrt{10}$.
The next analysis to obtain a similar gain in sensitivity must contain all $pp$ collision data collected up to end of 2022,
and then for the same increase again all data collected up to the end of 2038 must be included.

Therefore it can be seen that the increasing integrated luminosity of data collected by ATLAS will
allow for di-$b$-jet searches with increased sensitivity in the future,
although the time intervals between similar improvements of sensitivity become large.
In the remainder of this section other techniques to provide increases of sensitivity on a shorter time-scale are investigated.

%It should be noted that as the background estimate is data-driven, systematic uncertainties on the background also proportional to $\sqrt{L}$

%Figure~\ref{fig:fut-lumi} outlines the schedule of the LHC data-taking,
%showing the expected instantaneous and integrated luminosity against year.
%The grey blocks show the Long Shutdown (LS) periods of the LHC,
%in which no data such that the both the LHC accelerator and the ATLAS detector can be upgraded.
%Data-collected before LS1 used as centre-of-mass energy of 7 or 8 TeV for $pp$ collisions.
%All other data-collection periods used, or are scheduled to use, $pp$ collisions at a centre-of-mass energy of 13 TeV.
%
%%The data-collecting period before LS1, known as Run-1, used a centre-of-mass energy of 7 and 8 TeV for $pp$ collisions.
%%The data-collecting period between LS1 and LS2 is known as Run-2
%%and the data-collecting period scheduled between LS2 and LS3 is known as Run-3.
%%In Run-2 and Run-3 $pp$ collisions occur at centre-mass-energy of 13~\TeV  and have a similar intergrated luminosity.
%%In LS3 siginificant upgrades 
%%The data-collecting periods between LS1 and LS2 is called Run-2.
%
%The di-$b$-jet analyses presented in Chapters~\ref{sec:evt}-\ref{sec:lim} used data collected in 2015 and 2016.
%Specifically the \summer{} data-set analysis had an integrated luminosity of 13.3\ifb{} 
%and the \lm{} data-set had an integrated luminosity of 24.3\ifb.
%As described in Chapter~\ref{sec:evt}, the \lm{} data-set is soon to be published combined
%with the \hm{} data-set analysis, which contains 36.1\ifb{} of data and covers the same mass region as the \summer{} data-set.
%Therefore, with respect to the di-$b$-jet analyses published using the 2015 data-set, which contained 3.2\ifb{} of $pp$ collisions,
%the 2016 data-set has increased the sensitivity of the di-$b$-jet analysis by approximately a factor of $sqrt{10}$.
%
%,
%the addition of this data will increase sensitivity by a factor of $sqrt{10}$ with respect to the
%di-$b$-jet analyses  using the \lm{} and \hm{} data-set.
%In LS3, scheduled to begin in 2022, significant upgrades to the LHC accelerator and ATLAS detector are planned such that data can be taken at a higher instaneous luminosities,
%this is known as the Hi-Lumi LHC~\cite{}.
%Using the Hi-Lumi LHC, by 2038, it is expected that $\sim$3000\ifb{} of 13~TeV $pp$ collision of data will be collected,
%the addition of this data will again increase sensitivity by approximately a factor of $sqrt{10}$.
%
%Here it is shown the the increasing luminosity will allow for increase di-$b$-jet analyses with increased sensitivity.
%However, it can also be seen that the time periods between similar improvements of sensitivity become large.
%Therefore in the remainder of this Section other techniques to provide shorter term increases of sensitivity are also investigated.

\section{Combination of $b$-Tagging Categories}

The \summer{} data-set analysis presented in Chapters~\ref{sec:evt}-\ref{sec:lim} uses
two $b$-tag categories; the 2 $b$-tag and $\geq1~b$-tag category.
In both the search phase and limit setting phase the two categories were considered independently;
the former is used to search for a $Z'$ boson and the later is used to search for a $b^{*}$ quark.

However, a $Z'$ boson decaying two $b$-quarks can often be observed with one $b$-tag as a true $b$-jet may not be $b$-tagged.
Similarly, a $b^*$ quark can be observed with two $b$-tags as a gluon can split into two $b$-quarks which can be tagged.
The two features described above can be seen in Figure~\ref{fig:evt-ichep_acc}(b).

Hence, an improvement to the current analysis would be to consider three exclusive $b$-tagging categories;
where there are two jets that contain exactly 0, 1 or 2 $b$-tags.
Limits are then set on the benchmark models using a statistical combination of the three $b$-tagging categories.
This would allow for limits to be set on each model using the information from all three categories.

A di-$b$-jet search using a combination of the three $b$-tagging categories has been performed  by the CMS collaboration~\cite{dibjet-cms}.
The CMS analysis uses 8 TeV $pp$ collision data with an integrated luminosity of 19.6~\ifb{} in the mass region $\mjj>$~1.1~TeV.
Table~\ref{tab:fut-cmsComp} shows a comparison of the 95\% credibility-level observed upper mass limits set by the \summer{} and the CMS analysis.
The \summer{} data-set analysis sets a better upper mass limit on the $b^*$ quark than the CMS search;
likely due to the larger centre-of-mass energy used. The gain of combining categories is small for the $b^*$ as the $\geq1~b$-tag category is already used.
However, the CMS search is able to set a limit on the SSM $Z'$ boson, where the \summer{} data-set analysis cannot.
Therefore it is inferred that a combination of categories would lead to a significant improvement of limits set on the $Z'$ boson in future ATLAS di-$b$-jet analyses.

{\renewcommand{\arraystretch}{1.2}
\begin{table}[!htb]
\centering
\begin{tabular}{|c||c|c|c|c|}
  \hline
\multirow{2}{*}{Analysis} & \multirow{2}{*}{$\sqrt{s}$} & \multirow{2}{*}{Luminosity} & \multicolumn{2}{c|}{95\% CL Observed Upper Mass Limit} \\\cline{4-5}
                          &                             &                             & SSM $Z'$ boson              & $b^*$ quark             \\
\hline
ATLAS~\cite{dibjet-ichep_conf}& 13 TeV                  & 13.3 \ifb                   &       -                     &   2.3 TeV               \\
CMS~\cite{dibjet-cms}         & 7 TeV                   & 19.6 \ifb                   &       1.7 TeV               &   1.5 TeV               \\
\hline      
\end{tabular}
\caption[A comparison of the observed 95\% credibility level upper mass limits set by the \summer{} data-set analysis and a di-$b$-jet search performed by the CMS collaboration]
        {A comparison of the observed 95\% credibility level (CL) upper mass limits set on the SSM $Z'$ boson and $b^*$ quark by the \summer{} data-set analysis
          and a di-$b$-jet search performed by the CMS collaboration. A dash indicates that no limit was set.}
\label{tab:fut-cmsComp}
\end{table}}

\section{Improvement of $b$-Jet Trigger Efficiency Measurement}

In Chapter~\ref{sec:lim} it was shown that in the \lm{} data-set analysis there
is a large systematic uncertainty at high dijet mass due to the 
measurement of the $b$-jet trigger efficiency, the details of which are described in Chapter~\ref{sec:trig}.
Tables~\ref{tab:bTrig_jetSys}~and~\ref{tab:bTrig_eventEff} show that the largest sources of uncertainty
on the measurement of the $b$-jet trigger efficiency are non-$b$-jet impurities
and the extrapolation process required due to the low number of di-lepton $t\bar{t}$ events at high jet-\pT. 

Measurements of the offline $b$-tagging scale factors, described in Section~\ref{sec:obj-bjets_calib},
have similar sources of systematics uncertainties.
Techniques have been developed to reduce the systematic uncertainties in the offline $b$-tagging efficiency measurements~\cite{bj-bjets_calib_tech,obj-bjets_calib_plots};
such as the use of di-lepton $t\bar{t}$ events containing two electrons or two muons to increase statistics
and a Boosted Decision Tree (BDT) to increase the $b$-jet purity of jets selected.
Such techniques can be used to improve the $b$-jet trigger efficiency measurements.

A possible future development would be to combine the frameworks used by the $b$-jet trigger and offline $b$-tagging analyses such that
a combined offline plus $b$-jet trigger efficiency could be measured using the improved techniques from the offline $b$-tagging measurement described above.

\section{Signal plus Background Fits}

In Figure~\ref{fig:lim-lowmass_ssb_test} it was shown that, for the \lm{} data-set analysis,
there is a signal induced fit bias when the nominal background estimate is applied to a
spectrum injected with a dijet mass signal template of a SSM $Z'$ boson.
This could be a result of the requirement that higher order dijet fit functions
and more complex fitting models, such as SWiFt, are required to estimate the background from QCD dijet production at high luminosity.
The signal induced fit bias is removed when a technique employing a signal plus background fit is applied.
The signal plus background fit is not used in the search phase of the \lm{} analysis such that model independence can be maintained.

It is likely that a return background estimation models can be used in future di-$b$-jet analyses at high luminosities.
Therefore, to improve the sensitivity of the search phase in future analyses
a signal plus background fit should be considered, such that the signal induced biases are removed.
To reduce dependance of the search phase on any signal mode a large range of signal widths can be considered. 
A similar approach has been used at searches for resonances decaying into a pair of photons at ATLAS~\cite{fut-diphoton}.




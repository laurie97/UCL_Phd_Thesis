\addtocontents{toc}{\protect\pagebreak}
\chapter{Future Prospects of Di-$b$-Jet Searches}
\label{sec:fut}

This chapter will consider the future prospects of di-$b$-jet searches at ATLAS,
including a discussion of possible improvements and developments of the analyses presented in this thesis.

\section{Di-$b$-Jet Searches at Higher Luminosities} 

The LHC has been collecting 13~TeV $pp$ collision data since May 2015 and is scheduled to continue until 2038~\cite{fut-lhc-shedule}.
The di-$b$-jet searches presented in Chapters~\ref{sec:evt}-\ref{sec:lim} used 13~TeV $pp$ collision data collected in 2015 and 2016.
%From 2015-2022 it is expected that $pp$ collision data at a similar instantaneous luminosity as was collected in 2015 and 2016;
It is expected that the integrated luminosity of $pp$ collision data collected by the end of 2022 will be $\sim$300~\ifb{}~\cite{fut-lhc-shedule}.
After 2022, significant upgrades to the LHC accelerator and ATLAS detector are planned such that data can be taken at a higher instantaneous luminosity,
this is known as the High-Luminosity LHC.
The High-Luminosity LHC is expected to collect 14~\TeV{} $pp$ collision data with an integrated luminosity of $\sim$3000~\ifb{} by the end of 2038~\cite{fut-lhc-shedule}.

Table~\ref{tab:fut-lumi} summarises the integrated luminosity of the data-sets used by di-$b$-jet searches
at ATLAS and the expected integrated luminosities at the key points in the LHC schedule discussed above.
All di-$b$-jet searches at ATLAS use 13 TeV $pp$ collisions.
The table includes the \lm{} and \hm{} data-set analyses that are soon to be published together.

{\renewcommand{\arraystretch}{1.2}
\begin{table}[!htb]
\centering
\begin{tabular}{|c|r l|r l|}
  \hline
\textbf{End of Data}       &  \multicolumn{2}{c|}{\textbf{Integrated Luminosity}}         &   \multicolumn{2}{c|}{\textbf{Integrated Luminosity}}           \\
\textbf{Collection}        &  \multicolumn{2}{c|}{\textbf{using a Single Jet Trigger}}    &   \multicolumn{2}{c|}{\textbf{using a Double $b$-Jet Trigger}}  \\
\hline                                                                    
End of 2015       &  3.2 \ifb{} & \cite{dibjet-mori16_paper}            &  3.2 \ifb{}           & \cite{dibjet-lhcp_conf}          \\
July  2016        & 13.3 \ifb{} & \cite{dibjet-ichep_conf} (\summer{})  &  \multicolumn{2}{c|}{No analysis performed}              \\
End of 2016       & 36.1 \ifb{} & (\hm{})                               &  24.3 \ifb{}          & (\lm{})                          \\
End of 2022       & $\sim$ 300 \ifb{} &  \textit{(Projection)}          &  $\sim$ 300  \ifb{}   &  \textit{(Projection)}           \\
End of 2038       & $\sim$ 3000 \ifb{} & \textit{(Projection)}          &  $\sim$ 3000 \ifb{}   &  \textit{(Projection)}           \\
\hline
\end{tabular}
\caption[A summary of the integrated luminosity of data-sets used by the di-$b$-jet analyses performed at ATLAS and the expected integrated luminosities at key points in the LHC schedule]
        {A summary of the integrated luminosity of data-sets used by the di-$b$-jet analyses performed at ATLAS and the expected integrated luminosities at key points in the LHC schedule~\cite{fut-lhc-shedule}.
          All data-sets contain 13~TeV $pp$ collision data collected since May 2015,
          with the exception of the \lm{} data-set which is collected from March 2016.
          %Citations are given for published analyses.
          %Di-$b$-jet analyses of the \summer{} and \lm{} data-set are both presented together in Chapters~\ref{sec:evt}-\ref{sec:lim} of this thesis.
          %Di-$b$-jet analysis of the \hm{} data-set are soon to be published joint with the \lm{} data-set analysis.
        }
\label{tab:fut-lumi}
\end{table}}

The sensitivity of the di-$b$-jet analysis can be estimated as $S/\sqrt{B}$,
where $S$ and $B$ are the number of signal and background events passing the di-$b$-jet event selection in the mass region of the signal.
This approximation assumes that a perfect background estimation model is used and that there is no change in the systematic uncertainties used.
Therefore the estimated sensitivity of the di-$b$-jet analysis is proportional to the square root of the integrated luminosity.
Using this approximation and the values in Table~\ref{tab:fut-lumi}, it can be seen that the addition of data collected in 2016
increased the sensitivity of di-$b$-jet searches by a factor of $\sim\sqrt{10}$.
The next analysis to obtain a similar gain in sensitivity must contain all $pp$ collision data collected up to end of 2022,
and then for the same increase again all data collected up to the end of 2038 must be included.

Therefore it can be seen that the increasing integrated luminosity of data collected by ATLAS will
allow for di-$b$-jet searches with increased sensitivity in the future,
although the time intervals between similar improvements of sensitivity become large.
Furthermore, at the Hi-Lumi LHC it is likely that there might be other limitations,
for example $b$-tagging performance may decrease in the high track density environments expected at the Hi-Lumi LHC.
%% mu ~ 100 
Therefore, it is important to investigate other techniques to increase the sensitivity on a shorter time-scale.
%In the remainder of this section other techniques to provide increases of sensitivity on a shorter time-scale are investigated.

\section{Combination of $b$-Tagging Categories}

The \summer{} data-set analysis presented in Chapters~\ref{sec:evt}-\ref{sec:lim} uses
two $b$-tag categories; the 2 $b$-tag and $\geq1~b$-tag category.
The two categories are considered independently;
the former is used to search for a $Z'$ boson and the latter is used to search for a $b^{*}$ quark.
However, a $Z'$ boson can sometimes have only one $b$-tag as a true $b$-jet may not be $b$-tagged.
Similarly, a $b^*$ quark can have two $b$-tags as a gluon can split into two $b$-quarks which can be tagged.
The two features described above can be seen in Figure~\ref{fig:evt-ichep_acc}(b).

Hence, to increase the signal acceptance of the current analysis one could consider three exclusive $b$-tagging categories;
where there are two jets that contain exactly 0, 1 or 2 $b$-tags.
Limits are then set on the benchmark models using a statistical combination of the three $b$-tagging categories.
This would allow for limits to be set on each model using the information from all three categories.

A di-$b$-jet search using a combination of the three $b$-tagging categories has been performed by the CMS collaboration~\cite{dibjet-cms}.
The CMS analysis uses 8 TeV $pp$ collision data with an integrated luminosity of 19.6~\ifb{} in the mass region $\mjj>$~1.1~TeV.
Table~\ref{tab:fut-cmsComp} shows a comparison of the 95\% credibility-level observed upper mass limits set on the benchmark models
by the \summer{} data-set analysis and the CMS di-$b$-jet search;
the upper mass limit is the highest mass excluded.
The \summer{} data-set analysis sets a higher upper mass limit on the $b^*$ quark than the CMS search;
likely due to the larger centre-of-mass energy used.
The improvement from combining categories is smaller for the $b^*$ quark as the $\geq1~b$-tag category is already used by the \summer{} analysis.
However, the CMS search is able to set a limit on the SSM $Z'$ boson, where the \summer{} data-set analysis cannot.

It should be noted that the direct comparison of limits is not perfect as the CMS detector, object reconstruction and analysis structure is different
to those at ATLAS and different luminosities and centre-of-mass energies have been used.
That said, the comparison does suggest that a combination of categories could lead to a significant improvement
of the sensitivity to the $Z'$ boson signal models in future ATLAS di-$b$-jet analyses and should be investigated.

{\renewcommand{\arraystretch}{1.2}
\begin{table}[!htb]
\centering
\begin{tabular}{|c||c|c|c|c|}
  \hline
\multirow{2}{*}{\textbf{Analysis}} & \multirow{2}{*}{\textbf{$\sqrt{s}$}} &   \textbf{Integrated}                & \multicolumn{2}{c|}{\textbf{95\% CL Observed Upper Mass Limit}} \\\cline{4-5}
                          &                             &   \textbf{Luminosity}                & SSM $Z'$ boson              & $b^*$ quark             \\
\hline
ATLAS~\cite{dibjet-ichep_conf}& 13 TeV                  & 13.3 \ifb                   &       -                     &   2.3 TeV               \\
CMS~\cite{dibjet-cms}         & ~8 TeV                   & 19.6 \ifb                  &       1.7 TeV               &   1.5 TeV               \\
\hline      
\end{tabular}
\caption[A comparison of the observed 95\% credibility level upper mass limits set by the \summer{} data-set analysis and a di-$b$-jet search performed by the CMS collaboration]
        {A comparison of the observed 95\% credibility level (CL) upper mass limits set on the Sequential Standard Model $Z'$ boson and $b^*$ quark
         by the \summer{} data-set analysis at ATLAS
         and a di-$b$-jet search performed by the CMS collaboration~\cite{dibjet-cms}. 
          The upper mass limit represents the highest mass excluded by the analysis on the two benchmark models considered. A dash indicates that no limit was set.}
\label{tab:fut-cmsComp}
\end{table}}

\section{Improvement of $b$-Jet Trigger Efficiency Measurement}

In Section~\ref{sec:lim-full} it was shown that in the \lm{} data-set analysis there
is a large systematic uncertainty at high dijet mass due to the 
measurement of the $b$-jet trigger efficiency, the details of which are described in Chapter~\ref{sec:trig}.
Tables~\ref{tab:bTrig_jetSys}~and~\ref{tab:bTrig_eventEff} show that the largest sources of uncertainty
on the measurement of the $b$-jet trigger efficiency are caused by non-$b$-jet impurities
and the high-\pT{} extrapolation process required due to the low number of high-\pT{} jets in di-lepton $t\bar{t}$ events.

Techniques have been developed to reduce the same sources of systematic uncertainties in
measurements of the offline \footnote{Offline refers to objects reconstructed after events have passed the trigger at the data-processing level
  and online refers to reconstructed objects used in the trigger decision.
  From the definition in Section~\ref{sec:trig-bjet_eff}.}
$b$-tagging efficiency~\cite{obj-bjets_calib_tech,obj-bjets_calib_plots}, described in Section~\ref{sec:obj-bjets_calib}.
For example, a Boosted Decision Tree (BDT) is used to increase the $b$-jet purity of the selected jets
and di-lepton $t\bar{t}$ events containing two electrons or two muons are included to increase the number of di-lepton $t\bar{t}$ events.
Such techniques can be used to improve the $b$-jet trigger efficiency measurements.

A possible future development is to combine the frameworks used by the
$b$-jet trigger and offline $b$-tagging efficiency measurements.
This would allow for a combined offline plus online $b$-tagging efficiency measurement using
the improved techniques from the offline $b$-tagging measurement described above.

\section{Signal Plus Background Fit in the Search Phase}

In Figure~\ref{fig:lim-lowmass_ssb_test} it was shown that, for the \lm{} data-set analysis,
there is a signal induced fit bias when the nominal background estimate is applied to a
background-only dijet mass spectrum injected with a dijet mass signal template of a SSM $Z'$ boson.
This could be because higher order dijet fit functions
and more complex fitting models, such as SWiFt, are required to estimate the background from QCD dijet production at high luminosity.
The signal induced fit bias is removed when a technique employing a signal plus background fit is applied.
The signal plus background fit is not used in the search phase of the \lm{} analysis such that model independence can be maintained.

%It is likely that a return background estimation models can be used in future di-$b$-jet analyses at high luminosities.
Therefore, to improve the sensitivity of the search phase in future analyses
a signal plus background fit should be considered, such that the signal induced fit biases are removed.
To reduce the dependence of such a search phase on any signal model, a large range of signal widths should be considered. 
A similar approach has been used in a search for resonances decaying into a pair of photons at ATLAS~\cite{fut-diphoton}.

Furthermore, an analysis based around the signal plus background fit could be developed
such that same signal plus background fit is used in the search phase and limit setting phase.
This would simplify the analysis structure as the same background estimate would
be used in the search phase and limit setting phase.

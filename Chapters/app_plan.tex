%%%%%%%%%%%%%%%%%%%%%%%%%%%%%%%%%%%%%%%%%
%
%
%
\chapter{Contents Plan}

\noindent
A few notes from Laurie:\\
\indent\\
The aim is to be concise and to focus on what I used and did.
If I have not done or used something then I should not write it much detail about it.
Christian Gutchow did it in 100 pages including everything, this shows that it can be concise!

\section{Introduction}

\textbf{Status: Not Started}

\begin{itemize}
  \item{Introduce basics of analysis}
  \item{Explain why each section fits into larger picture}
\end{itemize}

\section{Theoretical Background}

\textbf{Status: Not Started}

\begin{itemize}
\item{The Standard Model (v. short)}
\item{Talk about QCD, background major background, part of standard model}
  \begin{itemize}[label={$-$}]
  \item{Major background, part of standard model}
  \item{Not an in-depth theoretical discussion, rather hints at why this is so large and smooth.}
  \end{itemize}     
\item{Why do we expect new physics (short)}
  \begin{itemize}[label={$-$}]
  \item{Dark Matter, SM generational structure}   
  \item{Why heavy flavour is interesting}
  \end{itemize}  
\item{Benchmark models - Z' and b*}
\end{itemize}

\section{ATLAS Detector}

\textbf{Status: First Draft Done}\\
\noindent
Still needs cleaning and response to comments

\begin{itemize}
\item{LHC and ATLAS}
  \begin{itemize}[label={$-$}]
  \item{Discuss CERN, LHC and ATLAS.}
  \item{Explanation of machine and luminosity plots in 2015+2016}
  \end{itemize}

\item{Inner Detector}
  \begin{itemize}[label={$-$}]
  \item{Pixel, SCT, TRT}
  \end{itemize}
\item{EM Cal.}
\item{Hadronic Cal.}
\item{Muon Chamber}  
\end{itemize}

\noindent
\\I see no reason to make this chapter too long.\\
Gutchow described detector in 3 pages!!, will not ask what Jiggins did it in.

\section{Object Reconstruction And Calibration}

\textbf{Status : First Draft In Progress}\\
\noindent
Done b-Tagging - calibration, doing jets.

\begin{itemize}

\item{Tracks}

\item{Jets}
  \begin{itemize}[label={$-$}]
  \item{Hadronic cluster reconstruction}
  \item{Jet Reconstruction anti-$k_T$}
  \item{Basics of calibration (JES, JER, BJES) description}
  \end{itemize}
  
\item{b-tagging}  
  \begin{itemize}[label={$-$}]
  \item{Algorithm descriptions}
    \begin{itemize}[label={$+$}]
    \item{Impact parameter based}
    \item{Secondary Vertex}
    \item{Jet Fitter}
    \item{Multi-variate}
    \end{itemize}
  \item{Calibration/performance (short): It would be nice to describe briefly and see results as it fits with b-Trigger}
  \item{bTagging: Validation in Dijet Events}
  \end{itemize}

\item{Leptons (short)}
  \begin{itemize}[label={$-$}]
  \item{Electron, Muon}
  \end{itemize}
\end{itemize}

  \noindent\\
  Validation in dijet events...\\
  

  \newpage
\section{Trigger}

\textbf{Status : First draft done}\\
\noindent
Needs a bit more cleaning and a bit of upmarketing



  \begin{itemize}
  \item{Intro to Trigger at ATLAS (why and how?)}
  \item{Jet Triggers}\\
    \url{https://twiki.cern.ch/twiki/bin/viewauth/Atlas/TrigJetOverview}\\%{https://twiki.cern.ch/twiki/bin/viewauth/Atlas/TrigJetOverview}
    \url{https://twiki.cern.ch/twiki/bin/view/Atlas/TrigJetMenuRunII}%{https://twiki.cern.ch/twiki/bin/view/Atlas/TrigJetMenuRunII}
    \begin{itemize}[label={$-$}]
    \item{Level 1}
    \item{HLT}
    \end{itemize}
  \item{b-Jet Triggers Description}
    \begin{itemize}[label={$-$}]
    \item{General decription}
    \item{Changes since Run-2}
    \end{itemize}
  \item{Efficiency of b-jet trigger measurement (long!) }
    \begin{itemize}[label={$-$}]
    \item{Strategy}
    \item{First plots and problem}
    \item{Investigation and Solution}
    \item{Measurement and Systematic Assignment}
    \item{Cross-checks: Electon/Muon overlap checks, reweighting of subleading}
    \end{itemize}
  \end{itemize}

  \noindent\\
  I think on this efficiency I have quite a bit of material, so is probably worth spending some time on as this was an important part of my work.\\

\section{Event Selection}

\textbf{Status: Not Started, Material from INT Note}

  \begin{itemize}
  \item{Jets : $p_{T}$ cut, eta cut}
  \item{$m_{jj}$ cut : Added complication here, we sometimes choose kinematic region based on fit quality}
  \item{$|y^*|$ cut (for analyses)}
  \item{Cleaning cuts ect.}
  \item{b-tagging requirements and efficiencies}
  \item{Overall event efficiency and event tagging}
  \item{VP1 displays}
  \end{itemize}

 Repeat for low mass and high mass...

\section{Background Estimation and Search Phase}

\textbf{Status: Not Started, Material from INT Note}


\begin{itemize}
  \item{High-mass flavour composition}
  \item{Fit Function Strategy and  bumpHunter Description}
  \item{Global fit at ICHEP - i.e. mass cut and spurious signal.}
  \item{SWiFt description (long!) }
  \item{SWiFt studies for 2017}
  \begin{itemize}[label={$-$}]
    \item{Fit quality and spurious signal, Signal injection}
  \end{itemize}
  \item{Search Phase results}
\end{itemize}
\noindent\\
Here there might be a challenge about how to combine studies from ICHEP and 2017.    

\section{Systematics and Limits Setting}


\textbf{Status: Not Started, Material from INT Note}

  \begin{itemize}
  \item{Systematics} 
    \begin{itemize}[label={$-$}]
    \item{Fit function chice and parameters}
    \item{Signal: Jets (JES/JER/BJES), b-Tagging and b-Trigger, Theoretical/pdf, Luminosity}
    \end{itemize}
  \item{Limit Setting}
    \begin{itemize}[label={$-$}]
    \item{Note Kate's thesis is somewhat of a bible on this so can reference this}
    \end{itemize}
  \item{Limits and discussion}
    \begin{itemize}[label={$-$}]
    \item{Maybe I can make a limit comparison plot with high/low mass and incl., would be useful}
    \end{itemize}
  \end{itemize}

\section{Looking Forward - What more can be done?}

\textbf{Status: Not started}

\begin{itemize}
  \item{Fit Function Options}
    \begin{itemize}[label={$-$}]
    \item{SWiFt development, other functions considered, (v. short as I don't work on it)}
    \end{itemize}
  \item{Combination of b-tagging channels}
    \begin{itemize}[label={$-$}]
    \item{Refer to CMS paper and the way they do it. We probably should have done this.}
    \end{itemize}
  \item{1 b-tag low mass}
    \begin{itemize}[label={$-$}]
    \item{See if appropriate trigger exists, could be done...}
    \end{itemize}
  \item{bTrigger efficiency.}
    \begin{itemize}[label={$-$}]
    \item{Optimise purity selection, bTrigger combined with offline b-tagging, (i.e. one systematic.) }
    \end{itemize}
  \end{itemize}
  
\newpage

\section{Rough list of things I have done}

Note from Laurie: Apologies, the stuff in italics is useful for me, and probably not for anyone else.

\section{2014/15 - High $p_{T}$ b-tagging}
\textit{See 2015\_09\_EndOfFirstYear.pdf and 15\_05\_TrackStudies.pdf  }
\begin{itemize}
\item{Studied pt distribution of tracks from different origin}
\item{Found a cut that would be able to increase selection of tracks from B-hadron}
\item{Suggested that this is taken into b-tagging algorihms}
  I think this is intersting, but I didn't really drive this through.
  I think some version of this was adopted, can  I refer to this.  
\end{itemize}

\section{2015 - Validation of b-tagging in dijet events}
\textit{See https://cds.cern.ch/record/2032461, 15\_09\_CTIDE.pdf, 2015\_09\_EndOfFirstYear.pdf }
\begin{itemize}
\item{Setup - selecting jets}
\item{Comparison of bunch of variables}
\item{Spot discrepancy in data in IP3D}
\item{What could be the problem}
  \begin{itemize}[label={$-$}]
  \item{IBL Geometry}
  \end{itemize}
\item{b-jet enhanced selection}
\item{New geometry comparisons}
  \subitem{\textit{See 16\_08\_newGeoComp.pdf}}
\end{itemize}

\section{2015 Di-b-jet - Moriond}
\begin{itemize}
\item{Background flavour fraction studies.}
\item{Plot flavour fraction.}
\item{Show robustness with respect to flavour fraction changes.}  
  \begin{itemize}[label={$-$}]
  \item{Extract flavour fractions}
  \item{Fit to individual flavour fractions}
  \item{Combine in various ways and re-fit}
  \end{itemize}
\subitem{(See~\textit{16\_03\_FlavourFit\_bumpHunter.pdf})}
\\
\item{VP1 Displays ( :] )}
\end{itemize}

\section{2015 low mass Di-b-jet - LHCP}

\begin{itemize}
\item{LHCP low mass MC studies. Difficult to do...}
\item{Flavour composition studies, various iterations.}
\item{Emulated trigger, emulated offline b-tagging, trigger from MC.}
\end{itemize}

\begin{itemize}
\item{Fitting studies with MC and fitting CR.}
\item{Changing fit CR, spurious signal}
\subitem{\textit{16\_05\_dibjet\_spuriousSignal\_EB2.pdf}}\\
\item{Effect on limits of any spurious signal.}
\item{Search for deficits in spectrum.}
\item{What happens if you play with paramter 2 of fit.}
\subitem{(\textit{16\_06\_dibjet\_S+B\_Check.pdf})}\\
\end{itemize}

\section{Half 2016 high mass Di-b-jet - ICHEP}

\begin{itemize}
\item{Fitting studies with MC.}
\item{Mass cut choice from MC fitting}
\item{Spurious signal check.}
\item{Background flavour fraction.}
\subitem{(\textit{16\_07\_mjjCut\_pValues\_INTnote.pdf}).}
\end{itemize}

\newpage
\section{2017 - bTrigger}
\textit{See bTriggerEfficiencies\_00-02-01.pdf}
\begin{itemize}
\item{Event selection}
\item{Derivation of efficiency}
\item{Investigation}
  \begin{itemize}[label={$-$}]
  \item{We spot problem, early plots period A-F}
  \item{Split into regions correctly}
  \item{Observation of dependance w.r.t online beamspot postion}
  \item{Suggestion of GRL creation of GRL}
    \subitem{\textit{See 17\_01\_Trig\_full.pdf}}
  \end{itemize}
\item{Jet-level Efficiency}
  \begin{itemize}[label={$-$}]
  \item{Purity Systematic}
  \item{Light Eff. Systematic}
  \item{Extrapolation to high pT}
  \item{Result}
  \end{itemize}
\item{Event-level correction}
  \begin{itemize}[label={$-$}]
  \item{Systematics and measurement}
  \item{Cross-checks, including on subleading jet.}
    \subitem{\textit{See 17\_01\_bTrigPres\_bPerf\_Eta.pdf}}
  \end{itemize}
\end{itemize}

\section{Full 2016 low mass Di-b-jet}

\begin{itemize}
\item{Analysis contact:}
  \begin{itemize}[label={$-$}]
  \item{Followed and reviewed all aspects of analysis closely,}
  \item{note editing, close interaction with paper editors ;),}
  \item{represenation of analysis in approval process.}
  \end{itemize}
\item{Event Selection:}
  \begin{itemize}[label={$-$}]
  \item{Trigger turn-on}
  \item{OP selection (with Nishu)}
  \item{y* selection (with Bing). }
  \end{itemize}
\item{Fit studies:}
  \begin{itemize}[label={$-$}]
  \item{Global fit fails}
  \item{SWiFt: Window selection procedure}
  \item{SWiFt: Spurious signal for various wHW and fit functions}
  \item{SWiFt: Signal injection tests}
  \end{itemize}
\end{itemize}

\section{Event Display}
\begin{itemize}
  \item{Was on call expert for 18 months}
\end{itemize}

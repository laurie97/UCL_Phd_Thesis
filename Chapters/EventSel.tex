\chapter{Di-$b$-jet Search: Outline and Event Selection}
\label{sec:evt}

In Section~\ref{sec:theo-bsm} it was shown that there are many deficiencies to the Standard Model
providing strong evidence that Beyond Standard Model (BSM) physics must exist; and motivations that this new physics could be at the TeV scale.
Many BSM models predict the existence of resonances decaying to one or two $b$-quarks that could be produced at the ATLAS experiement.
In Section~\ref{sec:theo-qcd-jets} it was described that free quarks would not be observed directly but instead as jets
and Section~\ref{sec:obj-jets}and~\ref{sec:obj-bjets} it was described how we can observe these jets and identify if they contain $b$-quarks.
Hence, so far we have motivated and broadly described the reconstruction techniques for
a search for resonances decaying to one or two $b$-jets,
an analysis that is called a di-$b$-jet search.

This chapter and the next two chapters will describe 
three such di-$b$-jet searches which are based on three different data-sets.
In this chapter, Section~\ref{sec:evt-outline} will give an outline of the
overall analysis strategy
and state how the analysis description has been split into three chapters.
Section~\ref{sec:evt-datasets}
describes the three different data-sets, motivating their use
and clearly labelling them for reference in future sections.

The remainder of the chapter will focus on event selection;
Section~\ref{sec:evt-s+b} will describe the signal and backgrounds
that we will consider in this analysis.
Section~\ref{sec:evt-sel} will describe
the event selection used to analyse each of the data-sets

\section{Analysis Outline}
\label{sec:evt-outline}

The strategy used for the di-$b$-jet analysis can be split up into broadly 3 parts,
which will be described in this chapter and the two following chapters.
A brief outline of these three parts is given here,
which will be substatiated in the following chapters.

\begin{itemize}[leftmargin=*]
\item\textbf{Di-$b$-jet Event Selection:}
  The first step is to select events that could come from a resonance decaying to one or two $b$-quarks;
  and as such we will require two high-momentum jets, where either at least one jet has been $b$-tagged been $b$-tagged.
  Full details and motivations of the selections applied are found in Section~\ref{sec:evt-sel}.
  The dominant background of this selection will be QCD multi-jet, which is discussed in Section~\ref{sec:evt-s+b}.
  \vspace{0.5em}
\item\textbf{Search Phase:}
  Once events have been selected the next part of the analysis aims to determine if there is
  any evidence of any resonances in the selected events, this step is known as the search phase.
  For this we will use the $m_{jj}$ spectrum, where $m_{jj}$ is the invariant mass of the two leading jets.
  It is known that the $m_{jj}$ distibution from QCD multi-jet will be smoothly falling,
  and as such we use a fit function to describe background.
  Any new particle will appear as a resonance or `bump' on the background; 
  Figure~\ref{fig:evt-dijet_schem} illustrates how such a resonance would look on top of a smoothly falling background.
  A model-independant search for such a feature is done using the bumpHunter algorithm.
  Chapter~\label{sec:bkg} contains a full description of the fitting and the bumpHunter strategy,
  along with the relevant tests of such a strategy and the results of the search-phase in the data-sets considered.  
  \vspace{0.5em}
\item\textbf{Limit Setting:}

  If, in the search phase stage of the analysis, no significant evidence of signal is
  found then we attempt to quantify the the level of confidence of that no signal has been found.
  We set 95\% confidence exclusion limits on our two benchmark signals (described in Section~\ref{sec:evt-s+b}
  and a generic gaussian signal.
  The limit-setting methodology, description of systematics considered
  and final limit results in the data sets considered is contained in Chapter~\ref{sec:lim}.


\end{itemize}







\section{Datasets}
\label{sec:evt-datasets}

\section{Signal and Backgrounds}
\label{sec:evt-s+b}

\section{Event Selection}
\label{sec:evt-sel}

\subsection{Jet Selection}
\label{sec:evt-sel-jet}

\subsection{Event Kinematics}
\label{sec:evt-sel-event}

\subsection{$b$-Tagging}
\label{sec:evt-sel-btag}

\subsection{Acceptance}
\label{sec:evt-sel-acc}

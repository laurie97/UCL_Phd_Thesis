\chapter{Introduction}
\label{sec:int}
\vspace{-1.5em}
The Standard Model is the current best description of the fundamental particles of the universe and their interactions.
However, inconsistencies within the Standard Model indicates that
there must be Beyond Standard Model (BSM) physics.
Many proposed BSM models predict the existence of new particles
which preferentially decay to a pair or $b$-quarks or a $b$-quark and a gluon.
The observation of such a particle would provide crucial experimental
evidence in the development of a more complete theory of particle physics.

%The high-energy $pp$ collisions produced by the LHC and observed
%by the ATLAS experiment provide an opportunity to search for
%BSM phyics in unexplored regions of phase space.

Searches for BSM resonances decaying to a pair of quarks or gluons
have been performed using the invariant mass distribution of pairs of hadronic jets created
by high-energy hadron collisions~\cite{theo-dijet_harris}, such searches are known as inclusive dijet searches.
Inclusive dijet searches have been performed in 13~TeV proton--proton collisions at the LHC~\cite{dijet-mori16_paper,dijet-mori17_paper,dijet-cms};
no evidence of a BSM resonance has yet been found.

The sensitivity of dijet searches to BSM models decaying to one or more $b$-quarks can be increased
using hadronic jets containing a $B$-hadron,
which leads to a significant reduction of Standard Model backgrounds.
Such searches are known as di-$b$-jet searches.
Di-$b$-jet searches have previously been performed by
the CDF collaboration using 1.8~TeV $p\bar{p}$ collisions at the Tevatron~\cite{dibjet-cdf}
and the CMS collaboration using 8~TeV $pp$ collisions at the LHC~\cite{dibjet-cms},
no evidence of a BSM resonance was found.

Di-$b$-jet searches have been performed using the ATLAS experiment
with higher-energy proton--proton collisions than any previous \mbox{di-$b$-jet} search,
representing an unprecedented opportunity to search for BSM resonances decaying to $b$-quarks.
This thesis presents two di-$b$-jet searches performed by using
13~TeV $pp$ collision data collected in 2015 and 2016 by the ATLAS detector.
A high-mass di-$b$-jet search probes the mass region 1.4~--~6~TeV using an integrated luminosity of 13.3~\ifb{};
the analysis has been published as a conference note~\cite{dibjet-ichep_conf}.
A low-mass di-$b$-jet search probes the mass region 0.6~--~1.5~TeV using an integrated luminosity of 24.3~\ifb{};
the analysis is soon to be published.

For the low-mass di-$b$-jet search in-time identification of $b$-jets is used to collect data,
this data-acquisition tool is known as the ATLAS $b$-jet trigger.
Therefore, for the low mass di-$b$-jet search a detailed understanding of the performance of the ATLAS $b$-jet trigger is required.
The measurement of the ATLAS $b$-jet trigger efficiency in 2016 data is also presented in this thesis.

\section{Structure of Thesis}

The thesis presents the di-$b$-jet searches in the following structure.

\noindent
Firstly, the theoretical and experimental background to the di-$b$-jet searches is discussed.
\begin{itemize}[leftmargin=*]
\item\textbf{Chapter 2} presents a description of the Standard Model,
  a summary of motivations for BSM physics and
  an outline of some BSM models that predict resonances
  decaying to one or more $b$-quarks.\vspace{0.5em}
\item\textbf{Chapter 3} presents a description of the LHC accelerator and the ATLAS detector.\vspace{0.5em}
\item\textbf{Chapter 4} presents the reconstructed physics objects used in di-$b$-jet searches. \vspace{0.5em}
\item\textbf{Chapter 5} presents a description of the triggers used in di-$b$-jet searches
  and the measurement~of the ATLAS $b$-jet trigger efficiency in 2016. \vspace{0.5em}
\end{itemize}
\noindent
Then, the two di-$b$-jet searches presented in this thesis are described in consecutive chapters.
\begin{itemize}[leftmargin=*]
\item\textbf{Chapter 5} presents an outline of the analysis strategy and the event selection used in the \mbox{di-$b$-jet} searches.\vspace{0.5em}
\item\textbf{Chapter 6} presents the search phase of the di-$b$-jet searches;
  which is a search for evidence of resonances in the di-$b$-jet events selected.
  The strategy and the results from the search phase for both di-$b$-jet searches are shown.\vspace{0.5em}
\item\textbf{Chapter 7} presents the limit setting phase of the di-$b$-jet searches.
  The strategy and the results of the limit setting phase for both di-$b$-jet searches are shown.\vspace{0.5em}
\end{itemize}
\noindent
  Finally, the work presented in this thesis summarised.
  \begin{itemize}[leftmargin=*]
\item\textbf{Chapter 8} presents an outlook of the future prospects of the di-$b$-jet searches.\vspace{0.5em}
\item\textbf{Chapter 9} presents the conclusions of the research presented in this thesis\vspace{0.5em}
\end{itemize}
\clearpage
\section{Personal Contributions}

In modern experimental particle physics most research is performed as part of large collaborations,
such that the technical complications of building, running and analysing the experiments can be shared amongst many.
One such of these collaborations is the ATLAS experiment at the LHC, comprised of over 3,000 physicists and engineers.

This thesis presents research performed between September 2014 and December 2017 carried out as part of the ATLAS collaboration.
To present the research in a complete form, the work must be presented within the context of the research carried out by the the ATLAS collaboration.
Furthermore, only the most significant contributions of the work are presented,
such that this thesis forms a coherent and consistent document without repetition of superseded results.

For clarity, this section summarises the author's personal contributions to the research activities of the ATLAS collaboration
and highlights where these are presented in the thesis.
Furthermore, all figures and tables that were not produced by the author are indicated using a citation in the caption.

\begin{itemize}[leftmargin=*]
\item\textbf{$b$-Tagging}:\\
  I was an active member of the $b$-tagging group between September 2014 and September 2015.
  I investigated improvements to $b$-tagging at high jet-\pT{} and
  performed the first data/simulation comparisons of $b$-tagging performance in dijet events using 13 TeV data collected between May and July 2015 by the ATLAS detector. \vspace{1em}
\item\textbf{Di-$b$-jet Search with \textit{Full15\_HighMass} data-set}:\\
  Between September 2015 and February 2016 I was a member of the analysis team that performed the first ever di-$b$-jet search at ATLAS.
  This analysis searched the mass range 1.2~--~5~TeV using 3.2~\ifb{} of 13~TeV $pp$ collision data collected in 2015 by the ATLAS detector.
  The analysis has been published here~\cite{dibjet-mori16_paper}. 
  I performed validated studies for the background estimation procedure. \vspace{1em}
\item\textbf{Di-$b$-Jet Search  with \textit{Full15\_LowMass} data-set}:\\
  Between February 2015 and June 2016 I was a member of the analysis team that performed the first ever di-$b$-jet search at ATLAS using a $b$-jet trigger.
  This analysis searched the mass range 0.6~--~1.2~TeV using 3.2~\ifb{} of 13~TeV $pp$ collision data collected in 2015 by the ATLAS detector.
  This analysis has been published as a conference note~\cite{dibjet-lhcp_conf}.
  I performed validated studies for the background estimation procedure. \vspace{1em}
    \newpage
\item\textbf{Di-$b$-Jet Search  with \summer{} data-set}:\\ 
  Between June 2015 and September 2016 I was a member of the analysis team for the \summer{} data-set analysis
  This analysis is presented in Chapters~\ref{sec:evt}-\ref{sec:lim}.
  The analysis has been published as a conference note~\cite{dibjet-ichep_conf}. 
  I was responsible for:
  \begin{itemize}
    \item Validating the background estimation and search phase (presented in Section~\ref{sec:bkg-summer})
    \item Selection of the mass range of the analysis (Section~\ref{sec:evt-sel})
    \item Creation of event displays (Section~\ref{sec:evt-sel})
  \end{itemize}
  \vspace{1em}
\item\textbf{$b$-Jet Trigger Efficiency Measurement  with \lm{}}:\\
  Between September 2016 and December 2017 as part of the $b$-jet trigger group,
  I was the lead analyser of the $b$-jet trigger efficiency measurement in 2016 data which is presented in Section~\ref{sec:trig-bjet_eff}.
  I was responsible for all aspects of the analysis, using a framework and strategy developed by John Alison.
  \vspace{1em}
\item\textbf{Di-$b$-Jet Analysis  with \lm{} data-set}:\\ 
  Between September 2016 and December 2017 I was a member of the analysis team for the \lm{} data-set analysis.
  The analysis is presented in Chapters~\ref{sec:evt}-\ref{sec:lim}.
  This analysis is soon to be published; therefore internal ATLAS documentation~\cite{dibjet-full_int} is cited to indicate work performed by other members of the analysis team.
  I was responsible for:
  \begin{itemize}
    \item All aspects of event selection, except $b$-tagging optimisation (Section~\ref{sec:evt-sel})
    %\item All aspects relating to the use of the $b$-jet trigger including application of scale factors and trigger emulation.
    \item Validation and results of the search phase (Section~\ref{sec:bkg-full})
    \item Adapting the data processing framework for the use of the $b$-jet trigger.
    \item Derivation of $b$-jet trigger and background systematic uncertainties,\\
      Creation of background templates used in limit-setting phase (both Section~\ref{sec:lim-full}).
    \item Representing the analysis to ATLAS collaboration as analysis contact.
  \end{itemize}
  \vspace{1em}
\item\textbf{Di-$b$-jet Analysis  with \hm{} data-set}: \\
  Between September 2016 and December 2017 I was a member of the analysis team for the \hm{} data-set analysis.
  This analysis is soon to be published together with the \lm{} data-set analysis.
  The analysis is not presented in this thesis.
  I contributed towards the validation of the background estimation and search phase.\vspace{1em}
\item\textbf{Event Display}: \\
  Between July 2015 and December 2016 I carried out maintenance of the {\sc ATLANTIS} Event Display used in the ATLAS control room
  and performed shifts as on-call `expert'.
\end{itemize}








\chapter{Di-$b$-jet Search: Limit Setting}
\label{sec:lim}

\section{Bayesian Limits}
\label{sec:lim-strat}

In Chapter~\ref{sec:bkg} it was shown that there was no evidence of signal in the dijet spectra considered.
Specifically, what was found was that the probability of obtaining a data-set with an excess similar to the one observed
under the assumption that there is new physics was above a certain threshold.
This lead to the conclusion that there is no evidence of BSM physics in the di-$b$-jet spectra.

However, it is also useful to further quantify this observation by asking the question
`what new physics models would have be observed in the di-$b$-jet spectrum if they were true?'
To consider this question, we invert the hypothesis test set out in the search phase;
instead let us assume that there is a new physics resonance
which produces $\mu$ di-$b$-jet events per \ifb in some known shape in~\mjj.
This signal is produced in addition to the QCD background,
which has been modelled by the background fit from the previous chapter.
This signal plus background model can be denoted with the hypothesis $H_\mu$.

Now let us consider this hypothesis in the context of the data,
which in this case is one of our di-$b$-spectra, and is denoted by $D$.
Let's say that in each~\mjj~bin, the model predicts
$s_i$ signal events, $b_i$ background events and $n_i$ events were observed.
For the hypothesis, $H_\mu$, the probability of producing a di-$b$-jet spectrum
such as the one we observed is known as the likelihood.
If no systematics are included the likelihood is simply given by
\begin{equation}
  \Like (H_\mu,D) = P(D \mid H_\mu) =  \Pi_i \frac{(s_i+b_i)^{n_i}~e^{-(s+b)_i}}{n_i!}
\end{equation}
Then, one can employ Bayes' theorem which states that
\begin{equation}
  P(A \mid B) = \frac{P(B \mid A) \, P(A)}{P(B)}
\end{equation}
to obtain the probability of hypothesis given the observed di-$b$-jet spectrum,
\begin{equation}
  P(H_\mu \mid D) = \frac{ P(D \mid H_\mu) \, \Pi( H_\mu ) }{ \Pi( D ) }
\end{equation}
The $\Pi(D)$ term does not depend on $\mu$ and as such can be considered as a normaliastion term.

The $\Pi( H_\mu )$ term  is called the signal prior
and gives the probability of $H_\mu$ before the experiment took place.
For this experiment we have chosen a flat signal prior
\footnote{Flat from $\mu$ = 0 to the value of $\mu$ where
likelihood has fallen to $10^{-5}$ of the optimal likelihood value.}
which represents that we are ignorant to the size of the signal before the experiment.

To accurately calculate a limit one must consider systematic uncertainties,
which can affect the models prediction of $s_i$ and $b_i$.
The systematics considered in this analysis are listed in Section~\ref{sec:lim-syst}.
The systematics are included in the Likelihood in the form of a set of nuissance parameters, $\vec{\theta}$,
such that the likelihood becomes a function of nuissance parameters
\begin{equation}
  \Like (H_\mu,D,\vec{\theta}) = P(D \mid H_\mu, \vec{\theta} ) 
\end{equation}
Then the nuissance parameters must be incorperated to equation~\ref{}.
A prior is introduced for each the nuissance parameters, given by $\Pi(\vec{\theta})$.
Then, the effect of the nuissnace parameter is propagated by integrating over
the nuissance parameters, which gives
\begin{equation}
  P(H_\mu \mid D) \propto \int d \vec{\theta} \, \Like (H_\mu, D, \vec{\theta} ) \, \Pi( H_\mu )  \, \Pi(\vec{\theta}) \, 
\end{equation}

Therefore one can calculate the Likelihoods for the data and
perform the integral over nuissance parameters for a range of values of $\mu$.
The value of $\mu$ for which  $P(H_\mu \mid D) =$ 0.05,
is the 95\% confidence level upper limit for the signal model
\footnote{ This is known as an upper limit as it is clear that if more signal events where present
  then $P(H_\mu \mid D) <$ 0.05 so can also be excluded,
  but similarly if less events where present then no 95\% exclusion could be applied.}.

In the di-$b$-jet analysis we will set limits using the benchmark signal model templates which were described in Section~\ref{}.
The limits will be presented as limits on cross-section x acceptance x efficiency,
which is equivalent definition as $\mu$ that was defined above.

Two upper limits are typical given for any search in particle physics.
The first is the observed limit, which is the limit using the observed di-$b$-spectra
as $D$ as described above.
The second is expected limits under the assumption that no signal is present, referred to as the expected limits.
To calculate these the limit setting procedure is performed where $D$ is replaced by pseudo-experiments created by applying Poisson fluctuations to the background model,
This process can be done for many pseudo-experiments and the median upper limit  found gives the expected limit
and the 68\% and 95\% percentiles give the 1 and 2 $\sigma$ error bands on the expected limit.




\section{Systematics}
\label{sec:lim-syst}

\section{Limits: 2016\_Summer}
\label{sec:lim-summer}

\section{Limits: 2016\_Full}
\label{sec:lim-full}

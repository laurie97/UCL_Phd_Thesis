\chapter{Di-$b$-jet Search: Search Phase}
\label{sec:bkg}

The role of the search phase is to identify
if there is any evidence of a resonance
in the~\mjj~spectra of the di-$b$-jet events events selected.
This is carred out in two parts;
firstly a background fit is used to estimate
the~\mjj~distribution of the QCD dijet background.
Then, the the difference between the data
and the background estimation is used 
to search 
for a significant excess that would be evidence
of a resonance.

In this chapter
I will describe
the details of the $m_{jj}$ spectra used in the analysis
(Section~\ref{sec:bkg-mjj}),
the background estimation strategy
(Section~\ref{sec:bkg-fit})
the technique used to search for excesses
(Section~\ref{sec:bkg-bh})
and then I will present the search phase
validation and results from each of the data sets.

\section{Invariant Dijet Mass Spectrum}
\label{sec:bkg-mjj}

The invariant mass spectrum is the number of events
with an invariant mass~\mjj~
of the leading and subleading jet.
The~\mjj~spectrum is analysed in a binned histogram,
the bin size is chosen to give a smooth spectrum
and to be larger than the size mass resolution of the detector.
The exact bins are chosen following the study found here~\cite{dijet-mori16_int}
and are shown in Appendix~\ref{app:dijet_bins}.

There is a different~\mjj~spectrum for each
$b$-tagging category considered,
and an independant search phase will be performed for both.

%Which is defined as;
%begin{equation}
% \mjj = \sqrt{ p^\mu_{L}^2 + p^\mu_{SL}^2 }
%end{equation}
%here $p^\mu_{L}$ and $p^\mu_{SL}^2$ are the four vectors of the leading and su

\noindent
The bins are:
\begin{verbatim}
203, 216, 229, 243, 257, 272, 287, 303, 319, 335, 352, 369,
387, 405, 424, 443, 462, 482, 500, 523, 544, 566, 588, 611, 
634, 657, 681, 705, 730, 755, 781, 807, 834, 861, 889, 917,
 946, 976, 1006, 1037, 1068, 1100, 1133, 1166, 1200, 1234, 
1269, 1305, 1341, 1378, 1416, 1454, 1493, 1533, 1573, 1614,
 1656, 1698, 1741, 1785, 1830, 1875, 1921, 1968, 2016, 2065, 
2114, 2164, 2215, 2267, 2320, 2374, 2429, 2485, 2542, 2600, 
2659,2719, 2780, 2842, 2905, 2969, 3034, 3100, 3167, 3235, 
3305, 3376, 3448,3521, 3596, 3672, 3749, 3827, 3907, 3988, 
4070, 4154, 4239, 4326, 4414, 4504, 4595, 4688, 4782, 4878, 
4975, 5074, 5175, 5277, 5381, 5487, 5595, 5705, 5817, 5931, 
6047, 6165, 6285, 6407, 6531, 6658, 6787, 6918, 7052, 7188, 
7326, 7467, 7610, 7756, 7904, 8055, 8208, 8364, 8523, 8685, 
8850, 9019, 9191, 9366, 9544, 9726, 9911, 10100, 10292, 
10488, 10688, 10892, 11100, 11312, 11528, 11748, 11972, 
12200, 12432, 12669, 12910, 13156
\end{verbatim}
\textbf{LM Fix: Move to appendix}

\section{Background Estimation}
\label{sec:bkg-fit}

Many analyses at ATLAS use Monte-Carlo simulation
to model backgrounds~\ref{obj-Hbb}.
However, simulation is not used to model the
background in the di-$b$-jet search due to three problems;
firstly it is difficult to produce Monte-Carlo simulation at high-enough statistical precision,
secondly there are large theoretical uncertainties for Monte-Carlo QCD
(such as PDF uncertainties and choice of renormalisation scale)
and finally there are large experimental uncertainties affecting
data-simulation comparisons (such as jet energy scale).

Instead the background is described using a smooth fit function.
This approach utilises the fact that the QCD dijet spectrum
is smoothly falling with respect to~\mjj,
as discussed in Section~\ref{sec:theo-qcd-dijet_features}.
Fit functions have been widely used
in a wide range of searches for resonances on smoothly falling backgrounds
including previous dijet and di-$b$-jet searches~\cite{dijet-mori16_paper,dibjet-mori16_paper}.

This approach gives two requirements on a fit function;
firstly the fit function must be able to describe the di-$b$-jet spectrum from QCD,
including the convolutions of any detector effects such as $b$-tagging.
Secondly,  the fit function used must be contrained enough
such that there is not a bias when signal is present in the di-$b$-jet spectrum.
As evidence of signal is found when the data diverges from the fit
such a bias would reduce the sensitivity to signal.
The fit functions considered in this analysis will be described in the following section.

For any given fit function, 
data is used to determine the parameters of the fit function.
This is done by minimising the negative log likelihood,
where the likelihood is calculated from comparing
the binned data to the fit function
under the assumption of poisson-like errors.

\subsection{Dijet Fit Functions}
\label{sec:bkg-bkg_func}

The di-$b$-jet mass spectrum is described by the dijet fit function:
\begin{equation}
  f(x)=p_1(1-x)^{p_2}(x)^{p_3+p_4\ln{x}+p_5(\ln{x})^{2}}
  %p_6(\ln{x})^{3}%},
\label{eqn:bkg-fit}
\end{equation}
where $p_i$ are fit parameters, and $x=m_{jj}/\sqrt{s}$.

Degrees of freedom can be removed to give a family of dijet fit functions which have a number of parameters ranging from 3 to 6.
The 3 parameter dijet fit function is defined by setting $p_{i} = 0$ for $i > 3$ in Equation~\ref{eqn:bkg-fit},
and the definition is equivalent for the 4 and 5 parameter dijet fit function.
There is in addition a 6 parameter dijet fit function where $x=m_{jj}/p_6$.

The choice of the function described by Equation~\ref{eqn:bkg-fit}
is motivated using a theoretical understanding of the QCD dijet production
and from previous dijet searches.
The three parameter equation has been used in dijet searches beginning with CDF~\cite{dijet-CDF_3par}
and the three components are motivated as follows:
the $p_1$ term gives the normalisation,
the $(1-x)^{p_2}$ term is a common parameterisation for the behaviour of the PDFs with the property of vanishing as $x$ approaches unity,
and the $x^{p_3}$ term is motivated by the $1/m_{kl}$ term in the matrix element (shown in Equation~\ref{eq:theo-qcd_dijet_xs}).
Additional parameters of $x^{p_4\ln{x}}$ and $x^{p_5\ln{x}^{2}}$ have been considered in dijet searches to give an adequete discription of the tail
when large mass ranges are considered~\cite{dijet-CDF_4par,dijet-mori16_int}.
Finally, the $x=m_{jj}/p_6$ term is added as an additional degree of freedom.
This function has been found to provide a satisfactory fit to the leading and next-to-leading-order QCD Monte-Carlo simulation.

Adding additional parameters to the 3-parameter dijet fit function may be required to describe the di-$b$-jet mass spectrum;
especially in large data-sets where small statistical errors reveal finer details of the QCD background shape
and large mass ranges where larger constraints are applied to the fit in each mass range.
However, additional parameters also allow for more flexibility in the background shape
which can cause a fit bias when signal is introduced.
Hence, there is an overall strategy to use the fewest parameters
that can adequetely describe the background,
such that we maximise sensitivity to signal.

\subsection{Wilks' Statistic}
\label{sec:bkg-bkg_wilks}

To decide whether a fit function has adequete number of parameters the Wilks' test statistic is used,
following previous iterations of both the inclusive and $b$-tagged dijet search~\cite{dijet-mori16_paper,dibjet-mori16_paper}.
\textbf{Add dibjet mori paper to references}.
The Wilks' test statistic tests the null hypothesis that the nominal fit function contains enough parameters to describe the data
by comparing the nominal to an alternate fit that has 1 extra parameter.
One can calculate the Wilks' test statistic, which is defined as $-\log{(\Lambda)}$, where $\Lambda$ is the likelihood ratio of the nominal and alternate function.
Using Wilks' theorem it is known that for a nested function, such as the function in Equation~\ref{eqn:bkg-fit},
the Wilks' test statistic will follow a $\chi^2$ distribution with 1 degree of freedom\textbf{LM Fix; wilks reference}.
Using this a $p$-value can be calculated for our null hypotheses that the nominal function has sufficient number of parameters.
Such a $p$-value will be referred to as the Wilks' $p$-value throughout this section.

%\subsection{Parameter Optimsation}
%\label{sec:bkg-bkg_param}

\section{BumpHunter Algorithm}
\label{sec:bkg-bh}

\section{Summer\_2016 Search Phase}
\label{sec:bkg-summer}

\subsection{Global Fit Strategy}
\label{sec:bkg-summer_global}

To select a fit function the following strategy is used with the final data-set.
The 3-parameter dijet function is used as the initial nominal function and hence the 4 parameter dijet function is the initial alternate.
Using the Wilks' statistic a $p$-value can be calculated,
if the $p$-value is less than 0.05, the nominal fit function is rejected and the alternative function would then become the nominal.
The process is iteratively run until a fit function is selected with a stable Wilks' $p$-value.

\subsection{Fit Tests: Background Only Sample}
\label{sec:bkg-summer_fitCR}

\subsection{Fit Tests: Mass Range Studies}
\label{sec:bkg-summer_range}

\subsection{Fit Tests: Spurious Signal}
\label{sec:bkg-summer_spusig}

\subsection{Search Phase}
\label{sec:bkg-summer_results}

\section{Full\_2016 Search Phase}
\label{sec:bkg-full}

SWiFt and ect...

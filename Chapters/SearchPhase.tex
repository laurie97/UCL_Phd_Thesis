\chapter{Di-$b$-jet Search: Search Phase}
\label{sec:bkg}

The role of the search phase is to identify
if there is signal present in the~\mjj~spectra
of the selected events.
This is carred out in two parts;
firstly a background fit is used to estimate
the~\mjj~distribution of the background QCD.
Then, using the difference between the data
and the background estimation we can search
for excesses that are evidence of a resonance
using the bumpHunter algorithm.

In this chapter I will describe
the fit functions used to describe the background
(Section~\ref{sec:bkg-fit})
the technique used to search for excesses
(Section~\ref{sec:bkg-bh})
and then I will present the search phase
validation and results from each of the data sets.

\section{Background Estimation}
\label{sec:bkg-fit}

Many analyses at ATLAS use Monte-Carlo simulation
to model backgrounds~\ref{obj-Hbb}.
However, in the case of modelling the background in the di-$b$-jet search
simulation is not used due to three problems;
firstly it is difficult to produce Monte-Carlo simulation at high-enough statistical precision,
secondly there are large theoretical uncertainties for Monte-Carlo QCD
(such as PDF uncertainties and choice of renormalisation scale)
and finally there are large experimental uncertainties affecting
data-simulation comparisons (such as jet energy scale).

Instead the background is described using a fit function
where one uses the data to determine the parameters of the fit function.
This approach utilises the fact that the QCD dijet spectrum
is smoothly falling with respect to~\mjj,
as discussed in Section~\ref{sec:theo-qcd-dijet_features}.
Fit functions have been widely used
in a wide range of searches for resonances on smoothly falling backgrounds
including previous dijet and di-$b$-jet searches~\cite{}
The fit function used must also be contrained enough
such that they will not be biased when signal is present in the dijet spectrum.
Such a bias would reduce the sensitivity to signal.
The fit functions considered in this analysis will be described below.

For any given fit function the parameters are optimised
by minimising the negative log likelihood where
the likelihood is calculated from comparing
the binned data \textbf{LM fix: binning discussion} to the fit function
under the assumption of poisson-like errors.

\subsection{Dijet Fit Functions}
\label{sec:bkg-bkg_func}

The di-$b$-jet mass spectrum is described by the di-jet fit function:
\begin{equation}
  f(x)=p_1(1-x)^{p_2}(x)^{p_3+p_4\ln{x}+p_5(\ln{x})^{2}}
  %p_6(\ln{x})^{3}%},
\label{eqn:bkg-fit}
\end{equation}
where $p_i$ are fit parameters, and $x=m_{jj}/\sqrt{s}$.

Degrees of freedom can be removed to give a family of dijet fit-functions which have an number of parameter ranging from 3 to 6.
The 3 parameter di-jet fit function is defined by setting $p_{i} = 0$ for $i > 3$ in Equation~\ref{eqn:bkg-fit},
and the definition is equivalent for the 4 and 5 parameter di-jet fit function.
There is in addition a 6 parameter di-jet fit function where $x=m_{jj}/p_6$.

The choice of the function described by Equation~\ref{eqn:bkg-fit}
is motivated using a theoretical understanding of the QCD dijet production
and from previous dijet searches.
The three parameter equation has been used in dijet searches beginning with CDF~\ref{dijet-CDF_3par}
and the three components are motivated as follows:
the $p_1$ term gives the normalisation,
the $(1-x)^{p_2}$ term is a common parameterisation for the behaviour of the PDFs with the property of vanishing as $x$ approaches unity,
and the $x^{p_3}$ term is motivated by the $1/m_{jj}$ term in the matrix element (shown in equation~\ref{}).
Additional parameters of $x^{p_4\ln{x}}$ and $x^{p_5\ln{x}}$ and have been considered in dijet searches to give an adequete discription of the tail
when large mass ranges are considered~\ref{dijet-CDF_4par,dijet-mori16_paper}.
Finally a $x=m_{jj}/p_6$ is added as an additional degree of freedom.
This function has been found to provide a satisfactory fit to the leading and next-to-leading-order QCD MC.\\

Adding additional parameters may be required to describe the di-$b$-jet mass spectrum;
especially at larger data-sets where small statistical errors reveal finer details of the QCD background shape
and large mass ranges where larger constraints are applied to the fit in each mass range.
However, additional parameters also allow for more flexibility in the background shape
which can cause a fit bias when signal is introduced.
Hence, the goal is to use the fewest parameters that can adequetely describe the background,
such that we maximise sensitivity to signal.

\subsection{Wilks' Statistic}
\label{sec:bkg-bkg_wilks}

To decide whether a fit function has adequete number of parameters the Wilks' test statistic is used,
as used in previous iterations of both the inclusive and $b$-tagged di-jet search~\ref{dijet-mori16_paper,dibjet-mori16_paper}.
The Wilks' test statistic tests the null hypothesis that the nominal fit function contains enough parameters to describe the data
by  comparing the nominal to an alternate fit that has 1 extra parameter.
One can calculate the Wilks' test statistic, which is defined as $-\log{(\Lambda)}$, where $\Lambda$ is the likelihood ratio of the nominal and alternate function.
Using Wilks' theorem it is known that for a nested function, such as the function in Equation~\ref{eqn:bkg-fit},
the Wilks' test statistic will follow a $\chi^2$ distribution with 1 degree of freedom\textbf{LM Fix; wilks reference}.
Using this a $p$-value can be calculated for our null hypotheses that the nominal function has sufficient number of parameters.
Such a $p$-value will be referred to as the Wilks' $p$-value throughout this section.

%\subsection{Parameter Optimsation}
%\label{sec:bkg-bkg_param}

\section{BumpHunter Algorithm}
\label{sec:bkg-bh}

\section{Summer\_2016 Search Phase}
\label{sec:bkg-summer}

\subsection{Global Fit Strategy}
\label{sec:bkg-summer_global}

The strategy employed then runs as follows:
the 3-parameter di-jet function is used as the initial nominal function and hence the 4 parameter di-jet function is the initial alternate.
Then, if the Wilks' $p$-value is less than 0.05, the nominal fit function is rejected and the alternative function would then become the nominal.
The process is iteratively run until a fit function is selected with a stable Wilks' $p$-value.

\subsection{Fit Tests: Background Only Sample}
\label{sec:bkg-summer_fitCR}

\subsection{Fit Tests: Mass Range Studies}
\label{sec:bkg-summer_range}

\subsection{Fit Tests: Spurious Signal}
\label{sec:bkg-summer_spusig}

\subsection{Search Phase}
\label{sec:bkg-summer_results}

\section{Full\_2016 Search Phase}
\label{sec:bkg-full}

SWiFt and ect...

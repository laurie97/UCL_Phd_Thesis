\chapter{An Incomplete Theory}
\label{sec:theo}

One of the great questions that humans have always tried to answer is
what is the world made up of and what are the rules that govern them?
Attempts at answering this question have lead to the
philosophical approach of \textit{`atomism'} by the ancient Greeks \cite{theo-atomism}
and the discovery of atomic structure by Ernest Rutherford \cite{theo-rutherford}.

The the current best answer to this question is known as the `Standard Model',
a mathematical description of a finite set of fundamental particles and their interactions.
%The Standard Model's abilty to describe data is formidable
%and as such it is the foundation of the field of particle physics 
However, it is known that this is not a complete theory and there must be
a deeper underlying theory that lies beyond the Standard Model.

This chapter firstly aims to describe the Standard Model and its key predicitions
with respect to the analyses within the context of this thesis.
Section~\ref{theo-sm} briefly describes the Standard Model and
Section~\ref{theo-qcd} describes QCD and jet formation,
and Section~\ref{theo-pdf} describes the theoretical understanding
of the structure of the proton, known as the parton density function.

Finally, Section~\ref{theo-bsm} will discuss physics Beyond the Standard Model (BSM);
specifically why it is thought that BSM physics is needed
and what are some of the possible models that one can search for.

\section{The Standard Model}
\label{theo-sm}

\subsection{Particles}

The Standard Model is made up of 18 fundamental particles,
which are grouped into three families of particles with similar properties;
known as quarks, leptons and bosons.

\begin{itemize}[leftmargin=*]
\item\textbf{Quarks:}
  Quarks are fermions, meaning they are spin-$\frac{1}{2}$ particles,
  that iteract with the strong force, which is described below.
  There are 6 different types of quarks, known as flavours, arranged in 3 generations.
  Table~\ref{tab:theo-sm_quarks} summarises the flavours of quark and their key properties,
  the values are taken from~\cite{obj-bjets_PDG}.
  For each quark there is also an anti-quark, which has identical mass and spin, but opposite charge and quantum numbers.
  %\end{itemize}
  {\renewcommand{\arraystretch}{1.5}
  \begin{table}[!ht]
  \begin{center}
    \begin{tabular}{|c||c|c|c|c|}
      \hline
    Quark Flavour & Symbol & Charge            &  Spin           &  Mass [GeV]\\
    \hline
    Up            &   $u$  &  $+\frac{2}{3}$   &  $\frac{1}{2}$  &  0.02\\
    Down          &   $d$  &  $-\frac{1}{3}$   &  $\frac{1}{2}$  &  0.05\\
    \hline                                                   
    Charm         &   $c$  &  $+\frac{2}{3}$   &  $\frac{1}{2}$  &  1.3 \\
    Strange       &   $s$  &  $-\frac{1}{3}$   &  $\frac{1}{2}$  &  0.1 \\
    \hline                                                      
    Top           &   $t$  &  $+\frac{2}{3}$   &  $\frac{1}{2}$  &  173  \\
    Bottom        &   $b$  &  $-\frac{1}{3}$   &  $\frac{1}{2}$  &  4.2  \\
    \hline  
  \end{tabular}
    \caption{The key properties of the 6 flavours of quark in the Standard Model,
    organised into the three generations of quarks.}
  \label{tab:theo-sm_quarks}
  \end{center}
  \end{table}}

\item\textbf{Leptons:}
  Leptons are fermions that, unlike the quarks, do not interact with the strong force
  There are 6 different types of leptons,
  arranged into  3 generations, each containing a charge $-1$ particle and a charge 0 neutrino.
  Table~\ref{tab:theo-sm_leptons} summarises the leptons and their key properties.
  The mass of the neutrino is not well known, but it is known to be non-zero and the sum of the three is less than a few eV ~\cite{theo-nu_mass}.
  As for quarks, for each lepton there is also an anti-lepton.
\newpage
  {\renewcommand{\arraystretch}{1.5}
  \begin{table}[!ht]
  \begin{center}
    \begin{tabular}{|c||c|c|c|c|}
      \hline
    Lepton            & Symbol        & Charge  &  Spin           &  Mass [GeV]\\
    \hline
    Electron          &   $e$         &  -1    &  $\frac{1}{2}$   &  \num{5.1e-4}\\
    Electron Neutrino &   $\nu_e$     &  0     &  $\frac{1}{2}$   &  -\\
    \hline                                   
    Muon              &   $\mu$       &  -1    &  $\frac{1}{2}$   &  0.106 \\
    Muon Neutrino     &   $\nu_{\mu}$  &  0     &  $\frac{1}{2}$   &  -\\
    \hline                                      
    Tau               &   $\tau$       &  -1   &  $\frac{1}{2}$   &  1.8\\
    Tau Neutrino      &   $\nu_{\tau}$  &  0    &  $\frac{1}{2}$   &  -\\
    \hline  
  \end{tabular}
    \caption{The 6 types of lepton in the Standard Model and their key properties,
    organised into the three generations of leptons. Neutrino masses are not well known. }
  \label{tab:theo-sm_leptons}
  \end{center}
  \end{table}}
 
\item\textbf{Bosons:}
  The bosons are the set of spin-0 or spin-1 particles in the SM,
  which act as the mediators of the forces that will be described below.
  Table~\ref{tab:theo-sm_bosons} summarises the bosons and their key properties.
  The bosons are their own anti-particle, with the exception of the $W^{+}$ and $W^{-}$
  which are each others anti-particle.

  {\renewcommand{\arraystretch}{1.5}
  \begin{table}[!ht]
  \begin{center}
    \begin{tabular}{|c||c|c|c|c|}
      \hline
    Boson            & Symbol        & Charge  &  Spin  &  Mass [GeV]\\
    \hline
    Photon           &   $\gamma$    &  0      &  1     &  0 \\
    W-boson          &   $W^{\pm}$    & $\pm$1  &  1     &  80 \\
    Z-boson          &   $Z_0$       &  0      &  1     &  91\\
    Gluon            &   $g$         &  0      &  1     &  0 \\
    Higgs Boson      &   $H$         &  0      &  0     &  125\\
    \hline  
  \end{tabular}
    \caption{The key properties of the bosons of the  Standard Model. }
  \label{tab:theo-sm_bosons}
  \end{center}
  \end{table}}
    
\end{itemize}

  
\subsection{Forces}

\begin{itemize}
\item\textbf{Electro-magnetic (EM):}
  
\item\textbf{Strong Force:}
  
\item\textbf{Weak Force:}

\item\textbf{Higgs Force:}
\end{itemize}
  

\section{QCD and Jets}
\label{theo-qcd}
  
\subsection{Matrix Element}
Why smooth...
\subsection{Parton shower}
\subsection{Hadronisation}
\subsection{A Special Case: $t\bar{t}$}

\section{Parton Density Function}
\label{theo-pdf}

\section{Beyond the Standard Model}
\label{theo-bsm}

\subsection{Why do we need BSM}
\subsection{Benchmark models}
\subsubsection{Z' Boson}
\subsubsection{Excited b* quark}

\chapter{An Incomplete Theory}
\label{sec:theo}

One of the great questions that humans have always tried to answer is
what are the fundamental building blocks of the universe and what are the rules that govern them?
Attempts at answering this question have ranged from
philosophical approach of \textit{`atomism'} by the ancient Greeks \cite{theo-atomism}
to the discovery of atomic structure by Ernest Rutherford \cite{theo-rutherford}.

The current best answer to this question is the \textit{`Standard Model of Particle Physics'},
a mathematical description of a finite set of fundamental particles and their interactions.
The Standard Model's ability to describe data is formidable
and as such it is the foundation of the field of particle physics. 
However, it is known that this is not a complete theory and there must be
a deeper underlying theory that lies beyond the Standard Model.

This chapter firstly aims to describe the Standard Model and its key predictions
with respect to the analyses within the context of this thesis.
Section~\ref{theo-sm} briefly describes the Standard Model and
Section~\ref{theo-qcd} outlines the QCD description of jet formation and dijet production in proton-proton collisions.
Then, Section~\ref{theo-bsm} will discuss physics Beyond the Standard Model (BSM);
specifically why it is thought that BSM physics is required
and what are some of the possible models that one can search for.

\newpage
\section{The Standard Model}
\label{theo-sm}

The Standard Model is a quantum field theory,
meaning that the theory describes a finite set of particles and their interactions in
terms of a set of fields.
The end product of the Standard Model is a prediction
of what will happen when any two particles in nature interact;
which in the context of a collider experiment means predicting what is the cross-section of any given interaction.

Section~\ref{theo-sm_particles} contains a description of the particles that make up the Standard Model 
and Section~\ref{theo-sm_forces} contains a description of the types of interactions between the particles, known as forces.

\subsection{Particles}
\label{theo-sm_particles}

There are 18 fundamental particles of the Standard Model,
where fundamental means that they are not composed of other constituent particles.
These particles are grouped into three families with similar properties;
known as quarks, leptons and bosons.
Details on the particles in the Standard Model is taken from~\cite{obj-bjets_PDG}, where a full description can be found.

\begin{itemize}[leftmargin=*]
\item\textbf{Quarks:}
  Quarks are fermions, meaning they are spin-$\frac{1}{2}$ particles,
  that interact with the strong force; a description of the strong force is in the next section.
  There are 6 different types of quarks, known as flavours, arranged in 3 generations.
  Table~\ref{tab:theo-sm_quarks} summarises the flavours of quark and their key properties.
  For each quark there is also an anti-quark, which has identical mass and spin, but opposite charge and quantum numbers.
  %\end{itemize}
  {\renewcommand{\arraystretch}{1.5}
  \begin{table}[!ht]
  \begin{center}
    \begin{tabular}{|c||c|c|c|c|}
      \hline
    Quark Flavour & Symbol & Charge            &  Spin           &  Mass [GeV]\\
    \hline
    Up            &   $u$  &  $+\frac{2}{3}$   &  $\frac{1}{2}$  &  0.02\\
    Down          &   $d$  &  $-\frac{1}{3}$   &  $\frac{1}{2}$  &  0.05\\
    \hline                                                   
    Charm         &   $c$  &  $+\frac{2}{3}$   &  $\frac{1}{2}$  &  1.3 \\
    Strange       &   $s$  &  $-\frac{1}{3}$   &  $\frac{1}{2}$  &  0.1 \\
    \hline                                                      
    Top           &   $t$  &  $+\frac{2}{3}$   &  $\frac{1}{2}$  &  173  \\
    Bottom        &   $b$  &  $-\frac{1}{3}$   &  $\frac{1}{2}$  &  4.2  \\
    \hline  
  \end{tabular}
    \caption{The key properties of the 6 flavours of quark in the Standard Model,
    organised into the three generations of quarks.}
  \label{tab:theo-sm_quarks}
  \end{center}
  \end{table}}

\item\textbf{Leptons:}
  Leptons are fermions that, unlike the quarks, do not interact with the strong force.
  There are 6 different types of leptons,
  arranged into  3 generations, each containing a charge $-1$ particle and a charge 0 neutrino.
  Table~\ref{tab:theo-sm_leptons} summarises the leptons and their key properties.
  The masses of the neutrinos are not well known, but they are known to be non-zero
  and the sum of the masses of the three flavours of neutrino is less than a few eV~\cite{theo-nu_mass}.
  For each lepton there is also an anti-lepton.
  
  {\renewcommand{\arraystretch}{1.5}
  \begin{table}[!ht]
  \begin{center}
    \begin{tabular}{|c||c|c|c|c|}
      \hline
    Lepton            & Symbol        & Charge  &  Spin           &  Mass [GeV]\\
    \hline
    Electron          &   $e$         &  -1    &  $\frac{1}{2}$   &  \num{5.1e-4}\\
    Electron Neutrino &   $\nu_e$     &  0     &  $\frac{1}{2}$   &  -\\
    \hline                                   
    Muon              &   $\mu$       &  -1    &  $\frac{1}{2}$   &  0.106 \\
    Muon Neutrino     &   $\nu_{\mu}$  &  0     &  $\frac{1}{2}$   &  -\\
    \hline                                      
    Tau               &   $\tau$       &  -1   &  $\frac{1}{2}$   &  1.8\\
    Tau Neutrino      &   $\nu_{\tau}$  &  0    &  $\frac{1}{2}$   &  -\\
    \hline  
  \end{tabular}
    \caption{The 6 types of lepton in the Standard Model and their key properties,
    organised into the three generations of leptons. Neutrino masses are not well known. }
  \label{tab:theo-sm_leptons}
  \end{center}
  \end{table}}
 
\item\textbf{Bosons:}
  The bosons are the set of spin-0 or spin-1 particles in the Standard Model,
  which act as the mediators of the forces that will be described below.
  Table~\ref{tab:theo-sm_bosons} summarises the bosons and their key properties.
  %The bosons are their own anti-particle, with the exception of the $W^{+}$ and $W^{-}$
  %which are each others anti-particle.

  {\renewcommand{\arraystretch}{1.5}
  \begin{table}[!ht]
  \begin{center}
    \begin{tabular}{|c||c|c|c|c|}
      \hline
    Boson            & Symbol        & Charge  &  Spin  &  Mass [GeV]\\
    \hline
    Photon           &   $\gamma$    &  0      &  1     &  0 \\
    W-boson          &   $W^{\pm}$    & $\pm$1  &  1     &  80 \\
    Z-boson          &   $Z_0$       &  0      &  1     &  91\\
    Gluon            &   $g$         &  0      &  1     &  0 \\
    Higgs Boson      &   $H$         &  0      &  0     &  125\\
    \hline  
  \end{tabular}
    \caption{The key properties of the bosons of the  Standard Model. }
  \label{tab:theo-sm_bosons}
  \end{center}
  \end{table}}
    
\end{itemize}

\subsection{Forces}
\label{theo-sm_forces}

The Standard Model combines three key theories in a
$SU(3)~\text{x}~SU(2)~\text{x}~U(1)$ gauge symmetry.
The first key theory is the electro-weak theory~\cite{theo-glashow};
this theory is based on mixing within the symmetry group $SU(2)~\text{x}~U(1)$
leading to three distinct interaction types grouped into two forces:
the electro-magnetic and weak forces.
The second is Quantum Chromodynamics (QCD)~\cite{theo-qcd} which describes the strong force.
Finally, the Brout-Englert-Higgs Mechanism~\cite{theo-be,theo-higgs} describes the origin of mass in the Standard Model.

\noindent
Each interaction is discussed in greater detail below:
\begin{itemize}[leftmargin=*]
\item\textbf{Electro-magnetic (EM):}

  The EM force is an interaction between charged particles and is mediated by the photon.
  The coupling is proportional to the product of the charges of the two particles
  multiplied by the EM coupling constant $\alpha_{EM}$, where $\alpha_{EM} \sim$ 1/137.\\ %\vspace{0.5em}

\item\textbf{Weak Force:}
  
  The weak force is composed of the two remaining interactions from electro-weak theory;
  the neutral current interaction and the charged current interaction.
  
  The neutral current interaction is mediated by the $Z_0$ boson, has a universal interaction to all fermions,
  and does not allow for flavour change within the interactions.

  The charged current interaction is mediated by the $W^+$ and $W^-$ boson, has a universal interaction with all fermions,
  and flavour changing interactions are allowed.
  Furthermore, due to the fact that the charged current interactions couples with weak eigenstates of fermions rather than
  their flavour eigenstates, the charged current interaction allows for interactions that change generation of the fermion's flavour.
  
  In the quark sector, the relative amplitudes of each flavour changing interactions is described by the CKM matrix;
  the structure of this matrix suppresses generational changing interactions,
  in particular those from the 3rd generation  are highly suppressed.
  This feature will prove important for identifying the presence of $b$-quarks at the ATLAS detector.
  Both interactions of the weak force are much weaker than the EM force due to the large masses of the mediating particles
  ($\text{Weak}/\text{EM} \sim 10^{-4}$).\\ %\vspace{0.5em}
  
\item\textbf{Strong Force:}

  Quantum Chromodynamics (QCD) is a theory described by a SU(3) gauge symmetry that describes the interactions between quarks and gluons.
  The symmetry leads to 3 colour charges: known as red, green and blue.
  If an object contains all three colour charges then it is a colour neutral object.
  The strong force is mediated by the gluon and interacts with particles that have colour charge; which are quarks and gluons.
  The fact that the gluon has colour charge means that the gluon is self interacting.
  QCD is important in terms of understanding hadronic jet formation and the production of the
  largest background in a dijet search, so further details can be found in section in Section~\ref{theo-qcd}.\\

\item\textbf{Higgs Mechanism:}
  The Higgs Mechanism \footnote{Also known as the Higgs-Englert-Brout mechanism}
  introduces an extra scalar field to the Standard Model
  and a Higgs potential given by the so-called `Mexican-hat potential'.
  This allows for spontaneous symmetry breaking which gives mass to the bosons of the Standard Model.
  In addition, a Yukawa coupling term between the scalar field and the fermions gives rise to the mass of the fermions
  \footnote{With the exception of the neutrinos, whose mass is not described by the Standard Model}.
  A final prediction of the Higgs mechanism was the existence of a spin-0 boson, known as the Higgs boson.
  The first observation of the Higgs Boson like object by the ATLAS~\cite{theo-higgs_atlas} and CMS~\cite{theo-higgs_cms} experiments
  in 2012 appears to confirm the Higgs mechanism, which would be a great triumph of the Standard Model.
\end{itemize}

\section{QCD: Hadronic Jet Formation and Dijet Production}
\label{theo-qcd}

As described above Quantum Chromodynamics (QCD) is a theory that describes the strong interaction between
quarks and gluons.
QCD therefore describes two elements that are critical to the analysis being presented in this thesis;
specifically the formation of hadronic jets and the production of dijet events through QCD in proton-proton collisions,
which will be the dominant background in the analysis.

This section will firstly describe renormalisation of QCD, which is important for understanding how QCD works,
and will then describe the process of hadronic jet formation and dijet production in hadron collisions.
Quarks and gluons can often fill similar roles in hadronic jet formation and dijet production, hence I will refer to them collectively as `partons' in this section.

\subsection{Renormalisation and the Running of $\alpha_S$}
\label{sec:theo-qcd_dijet_running}

For any calculation in QCD, or indeed any quantum field theory, one must consider the higher order loop diagrams;
for example for a simple gluon propagator there are additional first-order loops as shown in Figure~\ref{fig:theo-qcd_gluon}.
These additional loops lead to divergences in calculations of scattering events in QCD.

\begin{figure}[!hbt]
  \begin{center}
    \includegraphics[width=0.7\linewidth, angle=0]{figs/Theory/qcd_gluon_loop.pdf}
  \end{center}
  \caption[A schematic showing the gluon propagator with the additional first order loops.]
  {A schematic showing the gluon propagator with the additional first order loops~\cite{det-thesis_kate}.}
  \label{fig:theo-qcd_gluon}
\end{figure}

To avoid these divergences, there is a well accepted mathematical tool known as renormalisation,
where one effectively re-scales the fields in the Lagrangian.
This is done such that the divergences are removed
and one can perform calculations of QCD in a perturbative expansion.
This leads to a dependence of the strong coupling, $\alpha_S$, on the renormalisation scaled used, $\mu_R$,
an effect known as the running of $\alpha_S$.
To get an effective strength of the strong interaction in any given process,
$\mu_R$ is set close to the scale of the momentum transfer $Q$ of the process.
%$\alpha_S($\mu_R \sim Q^{2}$).
The running of $\alpha_S$ can be measured through experimental observation;
Figure~\ref{fig:theo-qcd_running} shows the measured values of
the strong coupling constant, $\alpha_S$ as a function of the energy scale, $Q$, in a range of experiments.

\begin{figure}[!hbt]
  \begin{center}
    \includegraphics[width=0.7\linewidth, angle=0]{figs/Theory/qcd_running.pdf}
  \end{center}
  \caption[Summary of measurements of $\alpha_S$ as a function of the energy scale $Q$.
    The respective degree of QCD perturbation theory used in the extraction of $\alpha_S$ is indicated in brackets
    (NLO: next-to-leading order; NNLO: next-to-next-to leading order; res. NNLO: NNLO matched with resummed next-to-leading logs; N3LO: next-to-NNLO).]
          {Summary of measurements of $\alpha_S$ as a function of the energy scale $Q$.
            The respective degree of QCD perturbation theory used in the extraction of $\alpha_S$ is indicated in brackets
            (NLO: next-to-leading order; NNLO: next-to-next-to leading order; res. NNLO: NNLO matched with resummed next-to-leading logs; N3LO: next-to-NNLO)~\cite{theo-qcd}.}
  \label{fig:theo-qcd_running}
\end{figure}

There are three features of Figure~\ref{fig:theo-qcd_running} that can be noted.
Firstly that the size of the coupling constant, $\alpha_S$, is large compared to the
$\alpha_{EM} \sim 1/137$;
this means that the strong force is stronger than the $EM$ force by at least an order of magnitude.
Secondly, at high-energies/low-distance scales the strong force becomes weak, such the quarks and gluons barely interact, this phenomenon is known as
\textit{`asymptotic freedom'}.
At these energy scales, perturbative expansions of QCD are possible.
Finally, at low-energies/large-distance scales the strong force is exceptionally strong.
As a result, if two interacting quarks become separated by a large distance then it becomes energetically favourable to
pair-produce a $q\bar{q}$ pairs from the vacuum until a colour neutral object can be formed.
Therefore quarks will never be observed in isolation but instead quarks form colour neutral hadrons, this feature of QCD is known as \textit{`confinement'}.

\subsection{Hadronic Jet Formation}
\label{sec:theo-qcd_jets}

It is common in hadronic colliders that a high-momentum quark or gluon will be produced in the final-state,
an example of this is dijet production, as described in Section~\ref{sec:theo-qcd_dijet}.
However, as described in Section~\ref{sec:theo-qcd_dijet_running},
the large values of $\alpha_S$ at large distance-scales require quark confinement, meaning that an isolated quark or gluon will not be observed.
Instead a stream of energetic, collimated hadrons will be formed, known as a hadronic jet.
Hadronic jet formation is described by two distinct processes; parton-shower and hadronisation.

\begin{itemize}[leftmargin=*]
  
\item\textbf{Parton Shower:}

  The high-energy final-state quark or gluon has a finite probability of splitting into a quark-gluon or quark-quark pair respectively.
  The resulting quarks and gluons will also undergo splitting to form more partons,
  which in turn can split. This process continues to form the parton shower.
  Due to relativistic effects, each splitting will generally be at a small opening angle in the lab-frame
  and as such the partons will be highly collimated in the direction of the initial parton.
  The parton shower process occurs at high energy such that the value of $\alpha_S$ is small
  and thus perturbative expansions of QCD can be used to perform calculations.
  However, at each step of the splitting the energy of the partons decreases
  and thus the value of $\alpha_S$ increases.\\
  
\item\textbf{Hadronisation:}
  
  When the energy scale becomes small
  \footnote{This is generally defined as small relative to the hadronic scale, $\Lambda$, which is typically a few hundred MeV},
  $\alpha_S$ becomes large such that the dominant QCD effect is quark confinement.
  Therefore, $q\bar{q}$ pairs are produced until the quarks resulting from the parton shower can form hadrons.
  The hadrons are colour neutral objects, meaning that stable hadrons that do not interact through QCD will be formed
  \footnote{Some unstable hadrons, such as a $\Delta^{++}$, may be initially formed in the process but these will decay rapidly through the strong interaction.
    In addition, some hadrons might not be stable under the weak interaction, such as a Kaon, but the time-scale of their decays will be much larger.}.
  The hadronisation process occurs at large values of $\alpha_S$ so cannot be calculated using perturbative expansions;
  to simulate hadronisation models such as the string model~\cite{theo-qcd_jet_string} and the
  cluster model~\cite{theo-qcd_jet_cluster} are used.

\end{itemize}
  
The end result of the hadronisation process is a set of collimated stable hadrons,
known as a hadronic jet, which can be observed in an experiment.
Note that our understanding of how one goes from an initial parton to a hadronic jet is model dependant,
for example there is a choice of hadronisation model.
Hence, in experiment we remove this dependence by defining a jet in terms of observables,
such that the experimental results are model-independent and results can be reinterpreted when improved models become available
\footnote{A good explanation of why model-independent jets is desirable is found here \url{https://www.theguardian.com/science/life-and-physics/2015/jan/10/this-is-not-a-measurement}}.
The details of the experimental definition of a hadronic jet is discussed in Section~\ref{sec:obj-jets}.

%\subsection{Parton shower}
%\subsection{Hadronisation}

\subsection{Dijet Production in $pp$ Collisions}
\label{sec:theo-qcd_dijet}


Dijet production is one of the most common process that occurs in any hadron collider.
The first step of dijet production in $pp$ colliders is the two protons interacting through QCD to give two quarks or gluons in the final state;
the relative frequency of this interaction is described by the hadronic cross-section, $\sigma_{had}$.
The free partons will then form hadronic jets through the processes described in Section~\ref{sec:theo-qcd_jets}.
As an example, Figure~\ref{fig:theo-qcd_dijet_feynman} shows the Feynman diagram of
dijet production in a proton-proton collision through one of the modes, the qg$\to$qg channel.

\begin{figure}[!hbt]
  \begin{center}
    \includegraphics[width=0.7\linewidth, angle=0]{figs/Theory/qcd_dijet_feynman.png}
  \end{center}
%  \caption[Three feynman diagrams illustrating the parton level scatter process in dijet production at the LHC.]
%          {Three feynman diagrams illustrating the parton level scatter process in dijet production at the LHC~\cite{theo-qcd_dijet_feynman}.}
  \caption[A Feynman diagram showing dijet production in a proton-proton collision through the qg$\to$qg channel.]
          {A Feynman diagram showing dijet production in a proton-proton collision through the qg$\to$qg channel~\cite{theo-qcd_dijet_feynman}.}
  \label{fig:theo-qcd_dijet_feynman}
\end{figure}

\subsubsection{Factorisation}

To calculate the hadronic cross-section, $\sigma_{had}$, in a proton-proton collision,
two elements are separated out in a process called factorisation.

The first element is the parton-level cross-section, $\hat{\sigma}$, which is the cross-section of
two partons from the proton ($p_a$ and $p_b$) scattering to give two final state partons ($p_i$ and $p_j$).
This is effectively the central part of the Feynman diagram in Figure~\ref{fig:theo-qcd_dijet_feynman}.

The second element is the Parton Density Functions (PDFs), $f_a(x_a)$, which described the probability of
a proton yielding a parton, $p_a$, with momentum fraction, $x_a$.
Momentum fraction is defined as the fraction of the protons total momentum that the parton is carrying, $x = p_{\text{parton}}/p_{\text{proton}}$.

The elements are combined to calculate the total $\sigma_{had}$;
\begin{equation}
%  \sigma(p_1p_2\to q/g_i q/g_j) = \int dx_1 dx_2 f_1(x_1,\mu^2_F)f_2(x_2\mu^2_F) \sigma(q/g_1,q/g_2, \alpha_s(\mu^2_R),Q^2/\mu^2_F,Q^2/\mu^2_R)
  \sigma_{had} = \sum_{a,b,i,j} \int dx_a dx_b f_a(x_a,Q^2)f_b(x_b,Q^2) \hat{\sigma}(p_a, p_b\to p_i p_j)
\end{equation}
where there is an integral over all possible values of momentum fractions $x_a$ and $x_b$,
a sum over all possible partons from the two protons labelled $a$ and $b$,
and a sum over all possible final-state partons labelled by $i$ and $j$.
$Q^2$ is the energy scale of the collision.

With the two elements separated we can discuss each separately.

\subsubsection{Parton-level Cross-Section}
\label{sec:theo-qcd_dijet_xs}

To describe the parton-level cross-section we must first define a few variables.
The first is the invariant mass of the outgoing partons, $m_{ij}$, which is given in terms of the four-momentum of the two partons by;
\begin{equation}
  m_{ij}^2 = (p^\mu_i + p^\mu_j)^2  
\end{equation}
\noindent
Then there are two related angular variables, $y^*$ and $\theta^*$,
defined in terms of $y_i$, the rapidity of the outgoing parton $p_i$;
\begin{equation}
  y^* = (\frac{y_i - y_j}{2}),
\end{equation}
\begin{equation}
  cos(\theta^*) = \tanh(y^*)
\end{equation}
\noindent
Finally the Mandelstam variables, generally used to describe a 2$\to$2 particle scatter event, are defined as 
\begin{equation}
  \hat{s} = m_{ij}^2, \hspace{3mm}  \hat{t} = -\hat{s}\hspace{1mm}(1 - \cos{\theta^*}), \hspace{3mm} \hat{u} = - \hat{s}\hspace{1mm}(1+\cos{\theta^*})
\end{equation}

\noindent
The parton-level cross-section of incoming partons $a$ and $b$ scattering to give
outgoing partons $i$ and $j$ is given in terms of the variables $\theta^*$ and $m_{ij}$~\cite{theo-dijet_harris};
\begin{equation}
  \frac{d\hat{\sigma}(p_a, p_b \to p_i p_j)}{dm_{ij}\hspace{1mm}d\cos{\theta^*}} = \frac{ \pi \alpha_s}{m_{ij}}\hspace{1mm} \text{S}(ab \to ij) \hspace{1mm} \frac{1}{1+\delta_{ij}}
  \label{eq:theo-qcd_dijet_xs}
\end{equation}
Where $\text{S}(ab \to ij)$ gives the process dependant kinematics of a $ab \to ij$  parton scatter.
$\text{S}(ab \to ij)$ for each process is described in Table~\ref{tab:theo-qcd_dijet_s}.

%  {\renewcommand{\arraystretch}{1.5}
%  \begin{table}[!ht]
%  \begin{center}
%    \begin{tabular}{|c|c|}
%      \hline
%      Subprocess         & $\text{S}(ab \to ij)$ \\
%    q1q2 - q1q2          &    4 s + u / 9t       \\
%    q1barq2 - q1barq2    &    4 s + u / 9t       \\
%    q1q1 - q1q1          &    4 s + u / 9t +  4 s + t / 9u
%    q1barq1 - q2barq2
%    
%    \hline  
%  \end{tabular}
%    \caption{The key properties of the 6 flavours of quark in the Standard Model,
%    organised into the three generations of quarks.}
%  \label{tab:theo-sm_quarks}
%  \end{center}
%  \end{table}}
%

\begin{table}[!hbt]
  \begin{center}
    \includegraphics[width=0.7\linewidth, angle=0]{figs/Theory/qcd_dijet_stable.pdf}
  \end{center}
  \caption[A table showing the process dependant part of the parton cross-section, $\text{S}(ab \to ij)$, for each of the processes in dijet production.]
  {A table showing the process dependant part of the parton cross-section, $\text{S}(ab \to ij)$, for each of the processes in dijet production. Taken from Table 1 of~\cite{theo-dijet_harris}.}
  \label{tab:theo-qcd_dijet_s}
\end{table}

\newpage
\subsubsection{Parton Density Functions}
\label{sec:theo-qcd_pdf}

Parton Density Functions (PDFs) describe effectively what is the probability of finding a parton $p_a$ in a proton $P_a$
for a given momentum fraction $x_a$ and energy scale, $Q$.
They have been studied extensively
by combining experimental scattering measurements,
particularly from deep inelastic scattering using $ep$ colliders such as HERA~\cite{theo-qcd_hera}.

Figure~\ref{fig:theo-qcd_pdf} shows the $x\hspace{0.3mm}F(x,Q^2)$ for a $Q^2$ of 10 and $10^4$ $\text{GeV}^2$
from the MMHT2014 PDF set~\cite{theo-qcd_pdf}.
The various colours lines represent the PDF for each of the different partons.

\begin{table}[!hbt]
  \begin{center}
    \includegraphics[width=1\linewidth, angle=0]{figs/Theory/qcd_pdf.pdf}
  \end{center}
  \caption[MMHT2014 NNLO PDFs at $Q^2$ = 10 $\text{GeV}^2$ and $Q^2$ = $10^4$ $\text{GeV}^2$, with associated 68\% confidence-level uncertainty bands.]
  {MMHT2014 NNLO PDFs at $Q^2$ = 10 $\text{GeV}^2$ and $Q^2$ = $10^4$ $\text{GeV}^2$, with associated 68\% confidence-level uncertainty bands~\cite{theo-qcd_pdf}.}
  \label{fig:theo-qcd_pdf}
\end{table}


One can note that as $x$ increases the values of the PDF for the quarks and gluons will fall smoothly.
This is particularly notable for the gluon which is the dominant contribution at low values of $x$.
The exception to this general behaviour is the valence quarks,
$u_v$ and $d_v$,
which are the quarks that are typically seen as the constituents of a proton and have a peak value around $x \sim \frac{1}{3}$.

\subsubsection{Features of the Hadronic Cross-section}

There are three important features that one can qualitatively describe about the dijet hadronic cross-section
from the two factorised elements shown in Section~\ref{sec:theo-qcd_dijet_xs}~and~\ref{sec:theo-qcd_pdf}.
These important features will have significance when forming the dijet search analysis strategy in
Chapters~\ref{bkg}\textit{Background estimation chapter} and~\ref{evt}\textit{Event selection chapter}.

\begin{itemize}[leftmargin=*]
\item\textbf{Large cross-section :}\\
  The strong coupling constant $\alpha_s$ is much larger than the other forces,
  meaning that the dijet cross-section is large.
  As a result dijet production through QCD is one of the most common events at hadron colliders
  and will be the strongly dominant background in any dijet search.\vspace{0.5em}
\item\textbf{Behaviour with respect to $m_{ij}$ :}\\
  It can be seen that hadronic cross-section will
  lead to a smooth and monotonically decreasing spectrum
  with respect to $m_{ij}$ as a result of three factors.
  Firstly the cross section has a $1/m_{ij}$ term.
  Secondly, as shown in Section~\ref{sec:theo-qcd_dijet_running},
  $\alpha_S$ will smoothly decrease with increasing $Q$, which in this case is linked to $m_{ij}$.
  Finally, as $m_{ij}$ increases then the momentum fraction of the proton, $x$, required to create
  the dijet event will also increase.
  As shown in Figure~\ref{fig:theo-qcd_pdf}, the parton density function of the quarks and the gluon
  is smoothly falling as $x$ increases, which will lead to falling behaviour in the hadronic cross-section.
  \vspace{0.5em}
\item\textbf{Behaviour with respect to $y^*$ :}\\
  In all but one of the $\text{S}(ab \to ij)$ terms shown in Table~\ref{tab:theo-qcd_dijet_s},
  we see that, due to the $t$-channel diagram, there is a $1/\hat{t}$ term that will become large when $\cos{\theta^*} \to 1$.
  Hence, we find that there is a larger dijet cross-section at large values of $\cos{\theta^*}$ and $y^*$.
  \vspace{0.5em}
\end{itemize}

Finally it should be noted that the above description of the dijet cross-section is
not a full description of the QCD interactions at hadron-hadron collisions.
Here I have only considered the tree-level diagrams;
one needs to consider higher orders of perturbative QCD to give a fuller description of dijet production.
Related to that issue is the problems of initial state and final state radiation, known as ISR and FSR respectively.
ISR is when an additional parton is radiated of the incoming parton where FSR is when an additional parton is radiated of the outgoing parton,
which can lead to additional jets in an event, creating a multi-jet event.

One should also consider the Underlying Event (UE) which effectively comprises of the remnants of the proton not used in the hard-scatter.
The UE will mostly be hadronic activity and as a result can lead to additional jets in the event, again giving us a multi-jet event.

\subsection{A Special Case: $t\bar{t}$}
\label{sec:theo-ttbar}

The top-quark is a special case when discussing the formation of jets from quarks
resulting in the unique topology of $t\bar{t}$, which is often exploited by analyses.

There are two theoretically motivated features of the top quark which are distinctive.
Firstly, due to the large suppression of decays from the 3rd generation in the CKM matrix,
the top quark decays to a $b$-quark and a $W$-boson with a branching ratio of close to~1.
Secondly, the top quark is much heavier than the bottom quark
meaning that the decay to a $b$-quark is very energetically favourable.
Therefore, the flavour changing weak decay occurs on a shorter time-scale than parton shower process
and thus the $W$-boson and hadronic jet from the $b$-quark will form separate observables
\footnote{If the top-quark has a large-$p_T$ then the resulting $W$-boson and jet will merge.}.

As in dijet production, $t\bar{t}$ pairs can be produced in proton-proton collisions through QCD interactions.
The two top quarks will decay into two $W$-bosons and two jets containing $b$-quarks.
One mode of $t\bar{t}$ decay is when one $W$ decays into a $e~\nu_e$ pair and the other into a $\mu~\nu_\mu$ pair.
This is known as a di-lepton $t\bar{t}$ event, a Feynman diagram showing an example of a di-lepton $t\bar{t}$ event is shown in
Figure~\ref{fig:theo-ttbar}\footnote{This figure shows the $q\bar{q}$ mode of $t\bar{t}$ production. It should be noted that the $gg$ mode is the dominant at the LHC}.

\begin{figure}[!hbt]
  \begin{center}
    \includegraphics[width=0.7\linewidth, angle=0]{figs/Theory/ttbar.pdf}
  \end{center}
  \caption[A Feynman diagram showing an example of a di-lepton $t\bar{t}$ event.]
  {A Fenian diagram showing an example of a di-lepton $t\bar{t}$ event~\cite{theo-ttbar_feyn}.}
  \label{fig:theo-ttbar}
\end{figure}

Di-lepton $t\bar{t}$ forms a distinct experimental signature.
Two different flavour leptons in an event signifies that there has likely been two separate weak-decays which would typically be suppressed,
but here the large mass of the top overcomes this suppression.
In addition we have two jets formed from $b$-quarks, which can be identified in a detector.
The distinct signature of di-lepton $t\bar{t}$ events and the fact that they always contain $b$-jets
means that this decay topology is often used to obtain a pure sample of $b$-jets, such as in Section~\ref{sec:obj-bjets_calib}~and~\ref{sec:trig-bjet_eff}. 


\section{Beyond the Standard Model}
\label{theo-bsm}

\subsection{Why do we need BSM}
\subsection{Benchmark models}
\subsubsection{$Z'$ Boson}
\subsubsection{Excited $b^*$ quark}

% -------- Packages --------

% This package just gives you a quick way to dump in some sample text.
% You can remove it -- it's just here for the examples.
\usepackage{blindtext}

% This package means empty pages (pages with no text) won't get stuff
%  like chapter names at the top of the page. It's mostly cosmetic.
\usepackage{emptypage}

% The graphicx package adds the \includegraphics command,
%  which is your basic command for adding a picture.
\usepackage{graphicx}

% This command is provided by the graphicx package, and 
%  controls the default dpi resolution of images you use.
%  72 is the default, but 300 is more normal, and 600 is
%  as good as you can expect to be able to get on normal paper.
%\pdfimageresolution=300


% The float package improves LaTeX's handling of floats,
%  and also adds the option to *force* LaTeX to put the float
%  HERE, with the [H] option to the float environment.
\usepackage{float}

\usepackage{placeins}

% The amsmath package enhances the various ways of including
%  maths, including adding the align environment for aligned
%  equations.
\usepackage{amsmath}

% Use these two packages together -- they define symbols
%  for e.g. units that you can use in both text and math mode.
\usepackage{gensymb}
\usepackage{textcomp}
% You may also want the units package for making little
%  fractions for unit specifications.
%\usepackage{units}


% The setspace package lets you use 1.5-sized or double line spacing.
\usepackage{setspace}
\setstretch{1.5}

% That just does body text -- if you want to expand *everything*,
%  including footnotes and tables, use this instead:
%\renewcommand{\baselinestretch}{1.5}


% PGFPlots is either a really clunky or really good way to add graphs
%  into your document, depending on your point of view.
% There's waaaaay too much information on using this to cover here,
%  so, you might want to start here:
%   http://pgfplots.sourceforge.net/
%  or here:
%   http://pgfplots.sourceforge.net/pgfplots.pdf
%\usepackage{pgfplots}
%\pgfplotsset{compat=1.3} % <- this fixed axis labels in the version I was using

% PGFPlotsTable can help you make tables a little more easily than
%  usual in LaTeX.
% If you're going to have to paste data in a lot, I'd suggest using it.
%  You might want to start with the manual, here:
%  http://pgfplots.sourceforge.net/pgfplotstable.pdf
%\usepackage{pgfplotstable}

% These settings are also recommended for using with pgfplotstable.
%\pgfplotstableset{
%	% these columns/<colname>/.style={<options>} things define a style
%	% which applies to <colname> only.
%	empty cells with={--}, % replace empty cells with '--'
%	every head row/.style={before row=\toprule,after row=\midrule},
%	every last row/.style={after row=\bottomrule}
%}


% The mhchem package provides chemistry formula typesetting commands
%  e.g. \ce{H2O}
%\usepackage[version=3]{mhchem}

% And the chemfig package gives a weird command for adding Lewis 
%  diagrams, for e.g. organic molecules
%\usepackage{chemfig}

% The linenumbers command from the lineno package adds line numbers
%  alongside your text that can be useful for discussing edits 
%  in drafts.
% Remove or comment out the command for proper versions.
%\usepackage[modulo]{lineno}
% \linenumbers 


% Alternatively, you can use the ifdraft package to let you add
%  commands that will only be used in draft versions
%\usepackage{ifdraft}

% For example, the following adds a watermark if the draft mode is on.
%\ifdraft{
%  \usepackage{draftwatermark}
%  \SetWatermarkText{\shortstack{\textsc{Draft Mode}\\ \strut \\ \strut \\ \strut}}
%  \SetWatermarkScale{0.5}
%  \SetWatermarkAngle{90}
%}


% The multirow package adds the option to make cells span 
%  rows in tables.
\usepackage{multirow}


% Subfig allows you to create figures within figures, to, for example,
%  make a single figure with 4 individually labeled and referenceable
%  sub-figures.
% It's quite fiddly to use, so check the documentation.
%\usepackage{subfig}

% The natbib package allows book-type citations commonly used in
%  longer works, and less commonly in science articles (IME).
% e.g. (Saucer et al., 1993) rather than [1]
% More details are here: http://merkel.zoneo.net/Latex/natbib.php
%\usepackage{natbib}
\usepackage{url}

% The bibentry package (along with the \nobibliography* command)
%  allows putting full reference lines inline.
%  See: 
%   http://tex.stackexchange.com/questions/2905/how-can-i-list-references-from-bibtex-file-in-line-with-commentary
\usepackage{bibentry} 

% The isorot package allows you to put things sideways 
%  (or indeed, at any angle) on a page.
% This can be useful for wide graphs or other figures.
%\usepackage{isorot}

% The caption package adds more options for caption formatting.
% This set-up makes hanging labels, makes the caption text smaller
%  than the body text, and makes the label bold.
% Highly recommended.
\usepackage[format=hang,font=small,labelfont=bf]{caption}
\usepackage{subcaption}

% If you're getting into defining your own commands, you might want
%  to check out the etoolbox package -- it defines a few commands
%  that can make it easier to make commands robust.
\usepackage{etoolbox}

% For lists
\usepackage{enumitem}
\setlist{nosep}

\usepackage{siunitx}

\DeclareSIUnit\mm{\milli\metre}
\DeclareSIUnit\km{\kilo\metre}


\DeclareSIUnit\micron{\micro\metre}
\DeclareSIUnit\mrad{\milli\rad}
\DeclareSIUnit\gauss{G}
\DeclareSIUnit\eVperc{\eV\per\clight}

\DeclareSIUnit\TeV{\teva\ev}
\DeclareSIUnit\GeV{\giga\ev}
\DeclareSIUnit\MeV{\mega\ev}

\DeclareSIUnit\nanobarn{\nano\barn}
\DeclareSIUnit\picobarn{\pico\barn}
\DeclareSIUnit\femtobarn{\femto\barn}
\DeclareSIUnit\attobarn{\atto\barn}
\DeclareSIUnit\zeptobarn{\zepto\barn}
\DeclareSIUnit\yoctobarn{\yocto\barn}

\DeclareSIUnit\nb{\nano\barn}
\DeclareSIUnit\pb{\pico\barn}
\DeclareSIUnit\fb{\femto\barn}
\DeclareSIUnit\ab{\atto\barn}
\DeclareSIUnit\zb{\zepto\barn}
\DeclareSIUnit\yb{\yocto\barn}

%%%%%% From ATLAS stuff
% +--------------------------------------------------------------------+
% |                                                                    |
% |  Useful things for proton-proton physics                           |
% |                                                                    |
% +--------------------------------------------------------------------+
%
\def\pt{\ensuremath{p_{\mathrm{T}}}} % Subscript roman not italic (EE)
\def\pT{\ensuremath{p_{\mathrm{T}}}} % Subscript roman not italic (EE)
\def\et{\ensuremath{E_{\mathrm{T}}}} % Subscript roman not italic (EE)
\def\eT{\ensuremath{E_{\mathrm{T}}}} % Subscript roman not italic (EE)
\def\ET{\ensuremath{E_{\mathrm{T}}}} % Subscript roman not italic (EE)
\def\HT{\ensuremath{H_{\mathrm{T}}}} % Subscript roman not italic (EE)
\def\ptsq{\ensuremath{p^2_{\mathrm{T}}}} % Fixed so it works correctly (EE)

\def\degr{\ensuremath{^\circ}} % Removed mbox - caused problems and not needed (EE)
\def\abseta{\ensuremath{|\eta|}}
\def\Hgg{\ensuremath{H\to\gamma\gamma}}
\def\mh{\ensuremath{m_h}}
\def\mW{\ensuremath{m_W}}
\def\mZ{\ensuremath{m_Z}}
\def\mH{\ensuremath{m_H}}
\def\mA{\ensuremath{m_A}}
\def\MET{\ensuremath{E_{\mathrm{T}}^{\mathrm{miss}}}} % Sub/superscript roman not italic (EE)
\def\met{\ensuremath{E_{\mathrm{T}}^{\mathrm{miss}}}} % Sub/superscript roman not italic (EE)
\def\Wjj{\ensuremath{W \rightarrow jj}}
\def\tjjb{\ensuremath{t \rightarrow jjb}}
\def\Hbb{\ensuremath{H \rightarrow b\bar b}}
\def\Zmm{\ensuremath{Z \rightarrow \mu\mu}}
\def\Zee{\ensuremath{Z \rightarrow ee}}
\def\Zll{\ensuremath{Z \rightarrow \ell\ell}}
\def\Wln{\ensuremath{W \rightarrow \ell\nu}}
\def\Wen{\ensuremath{W \rightarrow e\nu}}
\def\Wmn{\ensuremath{W \rightarrow \mu\nu}}
\def\Hllll{\ensuremath{H \rightarrow ZZ^{(*)} \rightarrow \mu\mu\mu\mu}}
\def\Hmmmm{\ensuremath{H \rightarrow \mu\mu\mu\mu}}
\def\Heeee{\ensuremath{H \rightarrow eeee}}
\def\Amm{\ensuremath{A \rightarrow \mu\mu}}
\def\Ztau{\ensuremath{Z \rightarrow \tau\tau}}
\def\Wtau{\ensuremath{W \rightarrow \tau\nu}}
\def\Atau{\ensuremath{A \rightarrow \tau\tau}}
\def\Htau{\ensuremath{H \rightarrow \tau\tau}}
\def\begL{10$^{31}$~cm$^{-2}$~s$^{-1}$}
\def\lowL{10$^{33}$~cm$^{-2}$~s$^{-1}$}
\def\highL{10$^{34}$~cm$^{-2}$~s$^{-1}$}
\newcommand{\EjetRec}{\ensuremath{E_{\mathrm{rec}}}} % Subscript roman not italic (EE)
\newcommand{\PjetRec}{\ensuremath{p_{\mathrm{rec}}}} % Subscript roman not italic (EE)
\newcommand{\EjetTru}{\ensuremath{E_{\mathrm{truth}}}} % Subscript roman not italic (EE)
\newcommand{\PjetTru}{\ensuremath{p_{\mathrm{truth}}}} % Subscript roman not italic (EE)
\newcommand{\EjetDM}{\ensuremath{E_{\mathrm{DM}}}} % Subscript roman not italic (EE)
\newcommand{\Rcone}{\ensuremath{R_{\mathrm{cone}}}} % Subscript roman not italic (EE)
%
% +--------------------------------------------------------------------+
% |                                                                    |
% |  Some useful units                                                 |
% |                                                                    |
% +--------------------------------------------------------------------+
%
\def\TeV{\ifmmode {\mathrm{\ Te\kern -0.1em V}}\else
                   \textrm{Te\kern -0.1em V}\fi}%
\def\GeV{\ifmmode {\mathrm{\ Ge\kern -0.1em V}}\else
                   \textrm{Ge\kern -0.1em V}\fi}%
\def\MeV{\ifmmode {\mathrm{\ Me\kern -0.1em V}}\else
                   \textrm{Me\kern -0.1em V}\fi}%
\def\keV{\ifmmode {\mathrm{\ ke\kern -0.1em V}}\else
                   \textrm{ke\kern -0.1em V}\fi}%
\def\eV{\ifmmode  {\mathrm{\ e\kern -0.1em V}}\else
                   \textrm{e\kern -0.1em V}\fi}%
\let\tev=\TeV
\let\gev=\GeV
\let\mev=\MeV
\let\kev=\keV
\let\ev=\eV

\def\TeVc{\ifmmode {\mathrm{\ Te\kern -0.1em V}/c}\else
                   {\textrm{Te\kern -0.1em V}/$c$}\fi}%
\def\GeVc{\ifmmode {\mathrm{\ Ge\kern -0.1em V}/c}\else
                   {\textrm{Ge\kern -0.1em V}/$c$}\fi}%
\def\MeVc{\ifmmode {\mathrm{\ Me\kern -0.1em V}/c}\else
                   {\textrm{Me\kern -0.1em V}/$c$}\fi}%
\def\keVc{\ifmmode {\mathrm{\ ke\kern -0.1em V}/c}\else
                   {\textrm{ke\kern -0.1em V}/$c$}\fi}%
\def\eVc{\ifmmode  {\mathrm{\ e\kern -0.1em V}/c}\else
                   {\textrm{e\kern -0.1em V}/$c$}\fi}%
\let\tevc=\TeVc
\let\gevc=\GeVc
\let\mevc=\MeVc
\let\kevc=\keVc
\let\evc=\eVc

\def\TeVcc{\ifmmode {\mathrm{\ Te\kern -0.1em V}/c^2}\else
                   {\textrm{Te\kern -0.1em V}/$c^2$}\fi}%
\def\GeVcc{\ifmmode {\mathrm{\ Ge\kern -0.1em V}/c^2}\else
                   {\textrm{Ge\kern -0.1em V}/$c^2$}\fi}%
\def\MeVcc{\ifmmode {\mathrm{\ Me\kern -0.1em V}/c^2}\else
                   {\textrm{Me\kern -0.1em V}/$c^2$}\fi}%
\def\keVcc{\ifmmode {\mathrm{\ ke\kern -0.1em V}/c^2}\else
                   {\textrm{ke\kern -0.1em V}/$c^2$}\fi}%
\def\eVcc{\ifmmode  {\mathrm{\ e\kern -0.1em V}/c^2}\else
                   {\textrm{e\kern -0.1em V}/$c^2$}\fi}%
\let\tevcc=\TeVcc
\let\gevcc=\GeVcc
\let\mevcc=\MeVcc
\let\kevcc=\keVcc
\let\evcc=\eVcc

\def\cm{\ifmmode  {\mathrm{\ cm}}\else
                   \textrm{~cm}\fi}%
%
\def\ifb{\mbox{fb$^{-1}$}}%  Inverse femtobarns.
\def\ipb{\mbox{pb$^{-1}$}}%  Inverse picobarns.
\def\inb{\mbox{nb$^{-1}$}}%  Inverse nanobarns.
%
\def\mass#1{\ensuremath{m_{#1#1}}}%  "\mass{\mu}" produces "msub{mumu}".
\def\twomass#1#2{\ensuremath{m_{#1#2}}}% 
%
\def\Ecm{\ensuremath{E_{\mathrm{cm}}}} % Subscript roman not italic (EE)
%
